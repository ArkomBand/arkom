% Nama Kelompok : 
% Kelas : D4 TI 1A
% Anggota : 
% 1. Harun   	1174027
% 2. Fahmi   	1174021
% 3. Kukuh		1174016
% 4. Izzah		1174013
% 5. Rizal		1174014
% 6. Lawimner	1174030





Artikel tentang informasi mengenai RAM

\section{Pengertian RAM}
RAM kepanjangan dari Random Access Memory yang biasa terdapat di HP,di Komputer dan di leptop.
RAM adalah sebuah tipe penyimpanan komputer yang isinya dapat diakses dalam waktu yang tetap tidak mempedulikan letak data tersebut dalam memori.
RAM juga bisa menjadi tempat penyimpanan data,tapi hal ini hanya bersifat sementara saja.
RAM atau Random Acces Memory sebagai Memori Utama . Ram juga penentu seberapa cepat PC menjalankan Aplikasi.
RAM biasanya berukuran 128 mb 256 mb 512 mb 1 gb 2 gb 4 gb 8 gb 16 gb.

\section{Fungsi Ram}

Fungsi Ram adalah untuk mempercepat pemprosesan data pada PC/Komputer. 
Semakin besarnya RAM yang dimiliki, 
semakin cepatl pula komputer tersebut.

Selain itu, RAM juga berfungsi sebagai mendia penyimpanan disaat komputer atau laptop dalam keadaan hidup, 
apabila laptop atau komputer dimatikan maka data yang tersimpan dalam ram akan hilang dan terhapus. 
Misalkan disaat kita mengetik dokumen di microsoft word kemudian kita tutup tanpa klik save, 
data yang anda ketik akan tersimpan di memori ram, dengan begitu anda dapat membuka dokumen tersebut melalui history terakhir atau melalui auto save.

\section{Struktur Ram}

Ram juga memiliki 4 struktur utama yaitu:
1.input storage : yang memiliki fungsi untuk menampung input yang dimasukkan melalui input
2.Progam storage : Yang memiliki Fungsi untuk menyimpan semua instruksi instruksi progam yang akan diakses
3.Working storage : Yang memiliki fungsi untuk menyimpan data yang akan diolah dan hasil pengolahan
4.Output Storage : Yang memiliki fungsi untuk menampung hasil akhir dari pengolahan data yang akan ditampilkan ke alat output

\section{Sejarah RAM}
Random Acces Memory atau biasa di sebut RAM di temukan oleh Robert Dennard.
Pertama kali dikenal pada tahun 60'an. Hanya saja saat itu memori semikonduktor belumlah populer karena harganya yang sangat mahal. Saat itu lebih lazim untuk menggunakan memori utama magnetic. Perusahaan semikonduktor seperti Intel memulai debutnya dengan memproduksi RAM, lebih tepatnya jenis DRAM.
Perkembangan Random Access Memory (RAM) ini sangat cepat sehingga beberapa ahli komputer pun turut melakukan pengklasifikasian dari evolusi RAM ini . 
Berikut perkembangan RAM dari masa ke masa , diantaranya :

1 . RAM (Random Access Memory) . Ditemukan oleh Robert Dennard dan diproduksi secara besar-besaran oleh perusahaan Intel pada tahun 1968 , 
jauh sebelum komputer ditemukan oleh IBM pada tahun 1981. Darisinilah awal perkembangan RAM bermula . Pada awal diciptakannya, 
RAM membutuhkan tegangan 5,0 Volt untuk dapat berjalan pada frekuensi 4,77MHz dengan membutuhkan akses memory (Access time) sekitar 200ns (1ns = 10-9detik) 

2.	DRAM.(Dynamic Random Access Memory) Pada tahun 1970, IBM menciptakan sebuah memori yang dinamakan DRAM. DRAM sendiri merupakan singkatan dari Dynamic Random Access Memory. Dinamakan Dynamic karena jenis memori ini bekerja pada setiap interval waktu tertentu, selalu memperbarui keabsahan informasi atau isinya. DRAM mempunyai frekuensi kerja yang bervariasi, yaitu antara 4,77MHz hingga 40MHz. 

3.	FPM RAM. Fast Page Mode Dynamic Random Access Memoery atau disingkat dengan FPM DRAM ditemukan sekitar tahun 1987 atau yang lebih sering di kenal dengan nama FPM. FPM ini memungkinkan untuk transfer data yang lebih cepat pada baris (row) yang sama dari jenis memori sebelumnya yaitu DRAM. FPM bekerja pada frekuensi 16MHz hingga 66MHz dengan membutuhkan access time sekitar 50ns. Selain itu FPM mampu mengolah transfer data (bandwidth) sebesar 188,71 Mega Bytes (MB) per detiknya.

4.	EDO RAM. Pada tahun 1995, diciptakanlah memori jenis Extended Data Output Dynamic Random Access Memory (EDO DRAM) merupakan penyempurnaan dari FPM. Memori EDO dapat mempersingkat lingkaran membacanya sehingga dapat meningkatkan kinerjanya sekitar 20\%. EDO mempunyai access time yang bermacam macam, mulai dari 70ns hingga 50ns dan bekerja  pada frekuensi 33MHz hingga 75MHz. Meskipun EDO merupakan penyempurnaan dari FPM RAM, namun keduanya tidak dapat dipasang secara bersamaan, karena adanya perbedaan kemampuan kinerja. EDO DRAM sepertinya banyak digunakan pada sistem yang berbasis Intel 486 dan kompatibel dengan intel Pentium generasi awal.
5.	SDRAM PC66.(Synchronous Dynamic Randomn Access Memory) Pada awal tahun 1996 hingga akhir tahun 1997 Menemukan Synchronous Dynamic Random Access Memory atau disingkat SDRAM.SDRAM ini kemudian lebih dikenal sebagai PC66 karena bekerja pada frekuensi bus 66MHz,RAM ini biasanya terdapat pada komputer pentium 2-3,dan dia memiliki sifat membutuhkan tenaga cukup besar untuk dapat bekerja secara optimal
