\section{Pengertian RAM}
RAM kepanjangan dari Random Access Memory yang biasa terdapat di HP,di Komputer dan di leptop.
RAM adalah sebuah tipe penyimpanan komputer yang isinya dapat diakses dalam waktu yang tetap tidak mempedulikan letak data tersebut dalam memori.
RAM juga bisa menjadi tempat penyimpanan data,tapi hal ini hanya bersifat sementara saja.
RAM atau Random Acces Memory sebagai Memori Utama . Ram juga penentu seberapa cepat PC menjalankan Aplikasi.
RAM biasanya berukuran 128 mb 256 mb 512 mb 1 gb 2 gb 4 gb 8 gb 16 gb.

\section{Fungsi Ram}

Fungsi Ram adalah untuk mempercepat pemprosesan data pada PC/Komputer. 
Semakin besarnya RAM yang dimiliki, 
semakin cepatl pula komputer tersebut.

Selain itu, RAM juga berfungsi sebagai mendia penyimpanan disaat komputer atau laptop dalam keadaan hidup, 
apabila laptop atau komputer dimatikan maka data yang tersimpan dalam ram akan hilang dan terhapus. 
Misalkan disaat kita mengetik dokumen di microsoft word kemudian kita tutup tanpa klik save, 
data yang anda ketik akan tersimpan di memori ram, dengan begitu anda dapat membuka dokumen tersebut melalui history terakhir atau melalui auto save.

\section{Sejarah RAM}
Random Acces Memory atau biasa di sebut RAM di temukan oleh Robert Dennard.









\section{Struktur Ram}

Ram juga memiliki 4 struktur utama yaitu:
1.input storage : yang memiliki fungsi untuk menampung input yang dimasukkan melalui input
2.Progam storage : Yang memiliki Fungsi untuk menyimpan semua instruksi instruksi progam yang akan diakses
3.Working storage : Yang memiliki fungsi untuk menyimpan data yang akan diolah dan hasil pengolahan
4.Output Storage : Yang memiliki fungsi untuk menampung hasil akhir dari pengolahan data yang akan ditampilkan ke alat output