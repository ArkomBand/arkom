% Nama Kelompok : Linux
% Kelas : D4 TI 1A
% 1. Kadek Diva Krishna Murti (1174006)
% 2. Duvan Silalahi (1174011)
% 3. Oniwaldus (1174005)
% 4. Choirul Anam (1174004)
% 5. Sri Rahayu (1174015)
% 6. Ilham Habibi (1174028)

\documentclass{article}
\begin{document}

Memori disebut juga sebagai memori fisik merupakan suatu istilah generik yang merujuk pada media penyimpanan data sementara pada komputer. Setiap program dan data yang sedang diproses oleh prosesor akan disimpan di dalam memori fisik. Data yang disimpan pada memori fisik bersifat sementara, karena data yang disimpan di dalamnya akan tersimpan selama komputer tersebut masih dialiri daya dengan kata lain, komputer itu masih dalam keadaan hidup. Ketika sebuah komputer dimatikan atau direset, data yang disimpan dalam memori fisik akan hilang. Oleh sebab itulah sebelum anda mematikan komputer, anda harus menyimpan semua data yang belum anda simpan ke media penyimpanan permanen umumnya berbasis disk, seperti hard disk atau floppy disk, sehingga pada saat komputer anda dihidupkan kembali data tersebut dapat dibuka kembali di lain kesempatan. Memori fisik pada umumnya diterapkan dalam bentuk Random Access Memory (RAM), yang bersifat dinamis (DRAM). Disebut Random Access adalah karena akses terhadap tempat-tempat di dalamnya dapat dilakukan secara acak atau random, bukan secara berurutan atau sekuensial. Meskipun demikian, kata random access dalam RAM ini sering menjadi salah kaprah. Sebagai perumpamaan, memori yang hanya dapat dibaca seperti Read Only Memory (ROM,) juga dapat diakses secara random, tetapi ia dibedakan dengan RAM karena ROM dapat menyimpan data tanpa kebutuhan daya dan tidak dapat ditulisi sewaktu-waktu. Selain itu, hard disk yang juga merupakan salah satu media penyimpanan juga dapat diakses secara acak, namun hardisk tidak digolongkan ke dalam Random Access.

\section{Sejarah Memori}
Perkembangan micro computer, atau sering disebut juga dengan nama PC (Personal Computer) yang sedemikian pesat tentunya tidak lepas dari kebutuhan manusia akan informasi yang harus diolah oleh PC serta tentu saja perkembangan teknologi, khususnya teknologi perangkat keras, perangkat lunak, serta fungsi atau algoritma yang digunakan dalam memproses informasi yang diolah tersebut.
Pada awal ditemukannya PC banyak orang menganggap PC sebagai barang yang mahal atau mewah, namun kini anggapan itu tidak berlaku lagi karena hampir semua orang sudah memilikinya. Bisa dikatakan, orang yang tidak mengenal komputer pada zaman sekarang akan dicap sebagai orang yang gagap teknologi. Jika pada masa itu PC yang diotaki oleh prosessor Intel 8088 hanya mampu berjalan dengan kemampuan kecepatan 4,77 MHz yang digunakan untuk menajalankan program pengolah kata dalam pembuatan dan editing dokumen, spreadsheet sederhana untuk mengerjakan pekerjaan akuntansi maupun bisnis, dan program database sederhana serta sedikit program pendidikan dan game yang juga masih sangat sederhana. Kini PC yang diotaki Intel Pentium 4 mampu berjalan dengan kecepatan 2GHz, bahkan baru - baru ini Intel Corp melalui ajang Intel Developer Forum-nya, telah menunjukkan demo prosessor Intel berkecepatan 3,5GHz Suatu penemuan teknologi yang cukup fantastis dan muktakhir. Namun perkembangan selanjutnya kemampuan PC tidak selalu ditentukan oleh perkembangan prosessor semata, bisa juga faktor lainnya, seperti teknologi chipset, memori, kartu VGA, perangkat media simpan, dan sebagainya. Semua perangkat saling berevolusi dan berkembang ke arah yang lebih baik untuk bersama - sama membangun suatu sistem PC yang tangguh. Perkembangan kemampuan prosessor yang begitu pesat tentunya harus diimbangi dengan peningkatan kemampuan memori. Memori dibutuhkan oleh prosessor sebagai penyimpan data atau informasi sekaligus sebagai penyimpan hasil dari perhitungan yang dilakukan oleh prosessor itu sendiri, sehingga kemampuan memori dalam mengelola data tersebut sangatlah penting. Percuma saja apabila kita memliki sebuah sistem PC dengan prosessor berkecepatan tinggi apabila tidak diimbangi dengan kemampuan memori yang sepadan. Ketidak tepatan perpaduan kemampuan prosessor dengan memori dapat menyebabkan inefisiensi bagi keduanya. Andaikan apabila kita mempunyai sebuah prosessor yang mampu mengelola arus data sebanyak 100 instruksi per detiknya, sementara kita memiliki memori dengan kemampuan menyalurkan data ke prosessor sebesar 50 instruksi per detiknya. Yang terjadi adalah sistem akan mengalami ketidakseimbangan yang disebabkan perbedaan kecepatan kerja antara prosessor dengan memori yang berarti prosessor harus menunggu data dari memori dan menyebabkan data yang seharusnya dapat dikerjakan dalam waktu 1 detik, menjadi 2 detik karena kemampuan memori yang terbatas. 

\section{Penggunaan memori}
Komponen utama dalam suatu sistem komputer adalah Arithmetic and Logic Unit (ALU), Control Circuitry, Storage Space dan piranti Input/Output. Tanpa adanya memori, sebuah komputer hanya akan berfungsi sebagai perangkat pemroses sinyal digital saja, contohnya kalkulator atau media player. Yang membuat sebuah komputer dapat disebut sebagai komputer multi-fungsi (general-purpose)  adalah kemampuan  dari memori untuk menyimpan data, instruksi serta informasi. Komputer merupakan sebuah piranti digital oleh karena itu, informasi yang disajikan oleh komputer yaitu menggunakan sistem bilangan biner atau binary. File yang berupa teks, angka, gambar, suara dan video akan dikonversikan menjadi sekumpulan bilangan biner atau binary digit atau disingkat bit. Sekumpulan bilangan biner dikenal dengan istilah BYTE, dimana  1 bita sama dengan 8 bit, 1 bit sama dengan 1 karakter, 1 kilobita sama dengan 1024 bita, dan bps sama dengan bit per second, 1 kbps sama dengan 1000 bps, 1 mbps sama dengan 1.000.000 bps. Semakin besar ukuran memorinya maka semakin banyak pula informasi yang dapat disimpan di dalam media penyimpanan komputer.

\section{Jenis - Jenis Memori}
Berikut ini beberapa jenis memori yang banyak digunakan pada saat ini sebagai berikut:

\begin{enumerate}

\item RAM (Random Acces Memory) adalah Memory tempat Penyimpanan sementara pada saat komputer di jalankan dan dapat di akses secara acak atau random. Fungsi dari RAM adalah mempercepat pemrosesan data pada komputer. Semakin tinggi jumlah RAM yang Anda miliki, semakin cepat pula kemampuan komputer Anda dalam mengeksekusi.
Jenis Memory RAM :

\begin{itemize}

\item EDORAM (Extended Data Out RAM)  
\item SDRAM (Synchronous Dynamic RAM)  
\item DDR SDRAM (Double Data Rate Synchronous Dynamic RAM) 
\item RDRAM (Rambus Dynamic RAM)

\end{itemize}

\item Registers adalah media penyimpan internal CPU yang digunakan saat proses pengolahan data. Memori ini bersifat sementara, biasanya digunakan untuk menyimpan data saat diolah ataupun data untuk pengolahan selanjutnya. Sistem dan bus yang menghubungkan komponen-komponen eksternal CPU dengan sistem lain, seperti memori utama serta piranti masukan atau keluaran dan juga menghubungkan komponen – komponen internal CPU dengan system lain, seperti Arimathics Logics Unit, Unit Control, dan Registers system koneksi dan bus tersebut disebut CPU Interconnections. \cite{junior2016evolusi}

\item Read Only Memory disingkat ROM merupakan memori yang tidak dapat dihapus isinya, hanya dapat dibaca, dan sudah diisi oleh pabrik pembuat komputer atau bisa dikatakan tidak bisa diprogram kembali. Sebagian perintah pada ROM akan dipindahkan ke RAM. Perintah yang ada di ROM antara lain, perintah untuk menampilkan pesan dilayar, perintah untuk membaca Sistem Operasi dari disk, dan perintah untuk mengecek semua peralatan yang ada di Unit Sistem.
Perkembangan ROM (Read Only Memory)
-  Programble ROM disingkat PROM merupakan ROM yang bisa diprogram kembali dengan catatan hanya bisa diprogram 1 x.
- Re-Programble ROM disingkat RPROM merupakan ROM yang bisa diprogram ulang sesuai dengan yang kita inginkan.
- Eraseble Programble ROM disingkat EPROM merupakan ROM yang dapat dihapus dan diprogram kembali tetapi cara penghapusannya dengan menggunakan Sinar Ultraviolet.
- Electrically Eraseble Programble ROM disingkat EEPROM merupakan ROM yang bisa diprogram dengan Teknik Elektronik. \cite{junior2016evolusi}

\item DRAM (Dynamic RAM) adalah jenis RAM yang secara berkala harus disegarkan oleh CPU agar data yang terkandung di dalamnya tidak hilang. DRAM merupakan salah satu tipe RAM yang terdapat dalam PC.
Compmentary Meta-Oxyde Semiconductor disingkat CMOS merupakan jenis chip yang memerlukan daya listrik dari baterai. Chip ini berisi memori 64-byte yang isinya dapat diganti. Chip ini biasanya mengatur berbagai pengaturan - pengaturan dasar yang terdapat 
pada perangkat komputer, seperti piranti yang digunakan untuk memuat sistem operasi dan termasuk pula tanggal dan jam sistem. CMOS merupakan bagian dari ROM.

\item Sychronous Dynamic RAM disingkat SDRAM adalah jenis RAM yang merupakan kelanjutan dari DRAM namun telah disnkronisasi oleh clock sistem dan memiliki kecepatan lebih tinggi daripada DRAM. DRAM ini cocok digunakan untuk sistem dengan bus yang memiliki kecepatan sampai 100 MHz.

\item Dual In-line Memory Module disingkatan DIMM dari  berkapasitas 168 pin, kedua belah modul memori ini aktif, setiap permukaan adalah 84 pin. Berbeda dengan SIMM yang berfungsi hanya pada sebelah modul saja. Mensuport 64 bit penghantaran data. SDRAM
(Synchronous DRAM) menggunakan DIMM dan merupakan penganti dari DRAM, FPM (fast Page Memory) dan EDO. SDRAM pengatur (synchronizes) memori supaya sama dengan CPU clock untuk pemindahan data yang lebih cepat. Terdapat dalam dua kecepatan yaitu 100MHz (PC100) dan 133MHz (PC133). DIMM 168 PIN. DIMM merupakan jenis RAM yang populer dan paling banyak terdapat di pasaran.

Cache memori berkapasitas terbatas, memori ini berkecepatan tinggi dan lebih mahal dibandingkan memory utama. Berada diantara memori utama dan register pemroses, berfungsi agar pemroses tidak langsung mengacu kepada memori utama tetapi di cache memory yang kecepatan aksesnya yang lebih tinggi, metode menggunakan cache memory ini

akan meningkatkan kinerja sistem. Cache memori merupakan slah satu tipe RAM tercepat yang pernah ada, dan digunakan oleh CPU, hard drive, dan beberapa pernah lainnya.

Magnetik Disk  Disk adalah piringan bundar yang terbuat dari bahan tertentu (logam atau plastik) dengan permukaan dilapisi bahan yang dapat di magnetisasi. Mekanisme baca atautulis menggunakan kepala baca atau tulis yang disebut head yang dimana merupakan kumparan pengkonduksi (conducting coil ). Desain fisiknya, head bersifat stasioner sedangkan piringan disk berputar sesuai kontrolnya. Disk memiliki dua metode layout data, yaitu  constant angular velocity dan multiple zoned recording. Disk diorganisasikan dalam bentuk berupa cincin – cincin
konsentrisyang disebut track. Tiap track pada disk dipisahkan oleh gap. Gap digunakan untuk mengurangi atau mencegah kesalahan penulisan maupun pembacaan yang disebabkan melesetnya head atau karena interferensi medan magnet. Sejumlah bit yang sama akan menempati track - track yang tersedia. Semakin dalam maka kerapatan dari disk akan bertambah besar. Biasanya










\end{enumerate}

\end{document}




