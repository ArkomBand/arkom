% Nama Kelompok : Linux
% Kelas : D4 TI 1A
% 1. Kadek Diva Krishna Murti (1174006)
% 2. Duvan Silalahi (1174011)
% 3. Oniwaldus (1174005)
% 4. Choirul Anam (1174004)
% 5. Sri Rahayu (1174015)
% 6. Ilham Habibi (1174028)

Memori disebut memori fisik merupakan istilah generik yang merujuk pada media penyimpanan data sementara pada komputer. Setiap program dan data yang sedang diproses oleh prosesor akan disimpan di dalam memori fisik. Data yang disimpan pada memori fisik bersifat sementara, karena data yang disimpan di dalamnya akan tersimpan selama komputer tersebut

masih dialiri daya (dengan kata lain, komputer itu masih hidup). Ketika komputer itu direset atau dimatikan data yang disimpan dalam memori fisik akan hilang. Oleh karena itulah sebelum anda mematikan komputer semua data yang belum anda simpan ke media penyimpanan permanen (umumnya berbasis disk, semacam hard disk atau floppy disk),

sehingga data tersebut dapat dibuka kembali di lain kesempatan. Memori fisik umumnya diimplementasikan dalam bentuk Random Access Memory (RAM), yang bersifat dinamis (DRAM). Disebut Random Access adalah karena akses terhadap tempat-tempat di dalamnya dapat dilakukan secara acak (random), bukan secara berurutan (sekuensial). Meskipun demikian, kata random access dalam RAM ini