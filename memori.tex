% Nama Kelompok : Linux
% Kelas : D4 TI 1A
% 1. Kadek Diva Krishna Murti (1174006)
% 2. Duvan Silalahi (1174011)
% 3. Oniwaldus (1174005)
% 4. Choirul Anam (1174004)
% 5. Sri Rahayu (1174015)
% 6. Ilham Habibi (1174028)

Memori disebut memori fisik merupakan istilah generik yang merujuk pada media penyimpanan data sementara pada komputer. Setiap program dan data yang sedang diproses oleh prosesor akan disimpan di dalam memori fisik. Data yang disimpan pada memori fisik bersifat sementara, karena data yang disimpan di dalamnya akan tersimpan selama komputer tersebut masih dialiri daya (dengan kata lain, komputer itu masih hidup).
Ketika komputer itu direset atau dimatikan data yang disimpan dalam memori fisik akan hilang. Oleh karena itulah sebelum anda mematikan komputer semua data yang belum anda simpan ke media penyimpanan permanen (umumnya berbasis disk, semacam hard disk atau floppy disk), sehingga data tersebut dapat dibuka kembali di lain kesempatan. Memori fisik umumnya diimplementasikan dalam bentuk Random Access Memory (RAM), yang bersifat dinamis (DRAM).
Disebut Random Access adalah karena akses terhadap tempat-tempat di dalamnya dapat dilakukan secara acak (random), bukan secara berurutan (sekuensial). Meskipun demikian, kata random access dalam RAM ini sering menjadi salah kaprah. Sebagai contoh, memori yang hanya dapat dibaca (ROM), juga dapat diakses secara random, tetapi ia dibedakan dengan RAM karena ROM dapat menyimpan data tanpa kebutuhan daya dan tidak dapat ditulisi sewaktu-waktu.
Selain itu, hard disk yang juga merupakan salah satu media penyimpanan juga dapat diakses secara lunak, serta fungsi atau algoritma yang digunakan dalam memproses informasi yang diolah tersebut Masih terbekas dalam ingatan kita akan perayaan 20 tahun PC yang jatuh pada bulan Agustus 2001 yang lalu, yang apabila kita cermati saat ini kita berada pada masa dimana PC telah menjadi bagian yang tidak dapat dipisahkan dari kehidupan kita.
Jika pada awal ditemukannya, PC masih dianggap sebagai barang mahal, kini hampir semua orang sudah memilikinya. Bisa dikatakan. orang yang tidak mengenal komputer akan dicap sebagai orang yang gagap teknologi. Jika pada masa itu PC yang diotaki oleh prosessor Intel 8088 hanya mampu berjalan dengan kemampuan kecepatan 4,77 MHz yang digunakan untuk menajalankan program pengolah kata dalam pembuatan dan editing dokumen, spreadsheet sederhana untuk mengerjakan pekerjaan akuntansi maupun bisnis, dan program database sederhana serta sedikit program pendidikan dan game yang juga masih sangat sederhana.
Kini PC yang diotaki Intel Pentium 4 mampu berlari dengan kecepatan 2GHz, bahkan baru - baru ini Intel Corp melalui ajang Intel Developer Forum-nya, telah menunjukkan demo prosessor Intel berkecepatan 3,5GHz Suatu penemuan teknologi yang cukup fantastis dan muktakhir. Namun perkembangan selanjutnya kemampuan PC tidak selalu ditentukan oleh perkembangan prosessor semata.
Masih faktor lainnya, seperti teknologi chipset, memori, kartu VGA, perangkat media simpan, dan sebagainya. Semua perangkat saling berevolusi dan berkembang ke arah yang lebih baik untuk bersama - sama membangun suatu sistem PC yang tangguh. Perkembangan kemampuan prosessor yang pesat tentunya harus diimbangi dengan peningkatan kemampuan memori. Sebagai penyimpan data atau informasi yang dibutuhkan oleh prosessor sekaligus sebagai penyimpan hasil dari perhitungan yang dilakukan oleh prosessor, kemampuan memori dalam mengelola data tersebut sangatlah penting.
Percuma saja sebuah sistem PC dengan prosessor berkecepatan tinggi apabila tidak diimbangi dengan kemampuan memori yang sepadan. Ketidak tepatan perpaduan kemampuan prosessor dengan memori dapat menyebabkan inefisiensi bagi keduanya. Anggap kita memiliki prosessor yang mampu mengelola arus data sebanyak 100 instruksi per detiknya, sementara kita memiliki memori dengan kemampuan menyalurkan data ke prosessor sebesar 50 instruksi per detiknya. Lalu apa yang terjadi? Sistem akan mengalami bottleneck. Prosessor harus menunggu data dari memori.
Instruksi yang seharusnya dapat dikerjakan dalam waktu 1 detik menjadi 2 detik karena kemampuan memori yang terbatas.
Penggunaan memori
Komponen utama dalam suatu sistem komputer adalah Arithmetic and Logic Unit (ALU), Control Circuitry, Storage Space dan piranti Input/Output. Tanpa memori, komputer hanya berfungsi sebagai piranti pemroses sinyal digital saja, contohnya kalkulator atau media player. Yang membuat sebuah komputer dapat disebut sebagai komputer multi-fungsi (general-purpose)  adalah kemampuan 