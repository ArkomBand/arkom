% Nama Kelompok : Linux
% Kelas : D4 TI 1A
% 1. Kadek Diva Krishna Murti (1174006)
% 2. Duvan Silalahi (1174011)
% 3. Oniwaldus (1174005)
% 4. Choirul Anam (1174004)
% 5. Sri Rahayu (1174015)
% 6. Ilham Habibi (1174028)

Memori disebut juga sebagai memori fisik merupakan suatu istilah generik yang merujuk pada media penyimpanan data sementara pada komputer. Setiap program dan data yang sedang diproses oleh prosesor akan disimpan di dalam memori fisik. Data yang disimpan pada memori fisik bersifat sementara, karena data yang disimpan di dalamnya akan tersimpan selama komputer tersebut masih dialiri daya dengan kata lain, komputer itu masih dalam keadaan hidup. Ketika sebuah komputer dimatikan atau direset, data yang disimpan dalam memori fisik akan hilang. Oleh sebab itulah sebelum anda mematikan komputer, anda harus menyimpan semua data yang belum anda simpan ke media penyimpanan permanen umumnya berbasis disk, seperti hard disk atau floppy disk, sehingga pada saat komputer anda dihidupkan kembali data tersebut dapat dibuka kembali di lain kesempatan. Memori fisik pada umumnya diterapkan dalam bentuk Random Access Memory (RAM), yang bersifat dinamis (DRAM). Disebut Random Access adalah karena akses terhadap tempat-tempat di dalamnya dapat dilakukan secara acak atau random, bukan secara berurutan atau sekuensial. Meskipun demikian, kata random access dalam RAM ini sering menjadi salah kaprah. Sebagai perumpamaan, memori yang hanya dapat dibaca seperti Read Only Memory (ROM,) juga dapat diakses secara random, tetapi ia dibedakan dengan RAM karena ROM dapat menyimpan data tanpa kebutuhan daya dan tidak dapat ditulisi sewaktu-waktu. Selain itu, hard disk yang juga merupakan salah satu media penyimpanan juga dapat diakses secara acak, namun hardisk tidak digolongkan ke dalam Random Access.

\section{Sejarah Memori}
Perkembangan micro computer, atau sering disebut juga dengan nama PC (Personal Computer) yang sedemikian pesat tentunya tidak lepas dari kebutuhan manusia akan informasi yang harus diolah oleh PC serta tentu saja perkembangan teknologi, khususnya teknologi perangkat keras, perangkat lunak, serta fungsi atau algoritma yang digunakan dalam memproses informasi yang diolah tersebut.
Pada awal ditemukannya PC banyak orang menganggap PC sebagai barang yang mahal atau mewah, namun kini anggapan itu tidak berlaku lagi karena hampir semua orang sudah memilikinya. Bisa dikatakan, orang yang tidak mengenal komputer pada zaman sekarang akan dicap sebagai orang yang gagap teknologi. Jika pada masa itu PC yang diotaki oleh prosessor Intel 8088 hanya mampu berjalan dengan kemampuan kecepatan 4,77 MHz yang digunakan untuk menajalankan program pengolah kata dalam pembuatan dan editing dokumen, spreadsheet sederhana untuk mengerjakan pekerjaan akuntansi maupun bisnis, dan program database sederhana serta sedikit program pendidikan dan game yang juga masih sangat sederhana. Kini PC yang diotaki Intel Pentium 4 mampu berjalan dengan kecepatan 2GHz, bahkan baru - baru ini Intel Corp melalui ajang Intel Developer Forum-nya, telah menunjukkan demo prosessor Intel berkecepatan 3,5GHz Suatu penemuan teknologi yang cukup fantastis dan muktakhir. Namun perkembangan selanjutnya kemampuan PC tidak selalu ditentukan oleh perkembangan prosessor semata, bisa juga faktor lainnya, seperti teknologi chipset, memori, kartu VGA, perangkat media simpan, dan sebagainya. Semua perangkat saling berevolusi dan berkembang ke arah yang lebih baik untuk bersama - sama membangun suatu sistem PC yang tangguh. Perkembangan kemampuan prosessor yang begitu pesat tentunya harus diimbangi dengan peningkatan kemampuan memori. Memori dibutuhkan oleh prosessor sebagai penyimpan data atau informasi sekaligus sebagai penyimpan hasil dari perhitungan yang dilakukan oleh prosessor itu sendiri, sehingga kemampuan memori dalam mengelola data tersebut sangatlah penting. Percuma saja apabila kita memliki sebuah sistem PC dengan prosessor berkecepatan tinggi apabila tidak diimbangi dengan kemampuan memori yang sepadan. Ketidak tepatan perpaduan kemampuan prosessor dengan memori dapat menyebabkan inefisiensi bagi keduanya. Andaikan apabila kita mempunyai sebuah prosessor yang mampu mengelola arus data sebanyak 100 instruksi per detiknya, sementara kita memiliki memori dengan kemampuan menyalurkan data ke prosessor sebesar 50 instruksi per detiknya. Yang terjadi adalah sistem akan mengalami ketidakseimbangan yang disebabkan perbedaan kecepatan kerja antara prosessor dengan memori yang berarti prosessor harus menunggu data dari memori dan menyebabkan data yang seharusnya dapat dikerjakan dalam waktu 1 detik, menjadi 2 detik karena kemampuan memori yang terbatas. 

\section{Penggunaan memori}
Komponen utama dalam suatu sistem komputer adalah Arithmetic and Logic Unit (ALU), Control Circuitry, Storage Space dan piranti Input/Output. Tanpa adanya memori, sebuah komputer hanya akan berfungsi sebagai perangkat pemroses sinyal digital saja, contohnya kalkulator atau media player. Yang membuat sebuah komputer dapat disebut sebagai komputer multi-fungsi (general-purpose)  adalah kemampuan  dari memori untuk menyimpan data, instruksi serta informasi. Komputer merupakan sebuah piranti digital oleh karena itu, informasi yang disajikan oleh komputer yaitu menggunakan sistem bilangan biner atau binary. File yang berupa teks, angka, gambar, suara dan video akan

