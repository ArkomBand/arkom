\section{ASCII}
\subsection{Prinsip-Prinsip Umum ASCII}
	Dalam ASCII dikenal juga Worldbet. Worldbet adalah versi ASCII dari  International Phonetic Alphabet (IPA) dengan tambahan luas simbol fonetik yang saat ini tidak ada di IPA. Worldbet ini dirancang untuk sejumlah besar bahasa termasuk Bahasa India, Asia, Afrika dan Eropa. Pertimbangan suara khusus di masing – masing bahasa ini mengarah pada prinsip bahwa setiap simbol dasar akan mewakili suara ucapan urutan waktu yang berbeda secara spektral. Setiap jenis / r / akan memiliki IPA yang terpisah, bukan r graphemic yang digunakan di beberapa label. Allophones seperti plorives aspirated akan memiliki simbol dasar terpisah dari plosives yang tidak diaspirasikan, mereka adalah fonemik dalam bahasa di pertanyaan, jika tidak mereka akan ditandai dengan menggunakan simbol dasar plus (diakritik). Begitu berbeda secara spektral atau temporer karena secara perseptual berbeda, ketika komponennya didengar dalam isolasi. Vokal digolongkan ke posisi posisi nominal. Hal ini diakui bahwa warna vokal rinci dapat bervariasi antara bahasa untuk vokal nominal yang sama, namun simbol yang terpisah hanya akan ditetapkan ketika perbedaan cukup besar untuk membentuk fonem yang berbeda.