

/subsection{Keuntungan}
1. Suara listrik masuk atau keluar dari kabel bisa dicegah.
2. Bentuk kabel termurah yang tersedia untuk keperluan jaringan.
3.Mudah untuk menangani dan menginstal.

/subsection{Kekurangan}
1. Deformasi: Kerentanan twisted pair terhadap interferensi elektromagnetik sangat bergantung pada skema twisted pair (kadang dipatenkan oleh produsen) tetap terjaga selama pemasangan. Akibatnya, kabel twisted pair biasanya memiliki persyaratan ketat untuk tegangan tarik maksimal serta radius tikungan minimum. Kerapihan kabel twisted pair ini membuat praktik pemasangan menjadi bagian penting untuk memastikan kinerja kabel.
2. Delay condong: pasangan yang berbeda dalam kabel memiliki penundaan yang berbeda, karena tingkat twist berbeda yang digunakan untuk meminimalkan crosstalk di antara pasangan. Hal ini dapat menurunkan kualitas gambar saat beberapa pasang digunakan untuk membawa komponen sinyal video. Kabel miring rendah tersedia untuk mengurangi masalah ini.
3. Ketidakseimbangan: perbedaan antara kedua kabel pada pasangan dapat menyebabkan kopling antara mode umum dan mode diferensial. Diferensial terhadap konversi mode umum menghasilkan arus mode umum yang dapat menyebabkan interferensi eksternal dan dapat menghasilkan sinyal mode umum pada pasangan lainnya. Mode umum untuk konversi mode diferensial dapat menghasilkan sinyal mode diferensial dari interferensi mode umum dari pasangan lain atau sumber eksternal. Ketidakseimbangan dapat disebabkan oleh asimetri antara dua konduktor pasangan satu sama lain dan dalam hubungan dengan kabel lain dan perisai. Beberapa sumber asimetri adalah perbedaan diameter konduktor dan ketebalan insulasi. Dalam jargon telepon, mode umum disebut longitudinal dan mode diferensial disebutmetalik.

/subsection{KARAKTERISTIK UTP}

Karakteristik UTP sangat bagus dan memudahkan untuk bekerja dengan, menginstal, memperluas dan memecahkan masalah dan kita akan melihat skema pengkabelan yang berbeda yang tersedia untuk UTP, bagaimana membuat kabel UTP langsung, peraturan untuk operasi yang aman dan banyak hal keren lainnya!
Kategori 1/2/3/4/5/6/7 - spesifikasi untuk jenis kawat tembaga (kebanyakan kawat telepon dan jaringan adalah tembaga) dan jack. Angka (1, 3, 5, dll) mengacu pada revisi spesifikasi dan secara praktis mengacu pada jumlah tikungan di dalam kawat (atau kualitas koneksi dalam jack).

CAT1 biasanya digunakan untuk kabel telepon. Jenis kabel ini tidak mampu mendukung lalu lintas jaringan komputer dan tidak terpelintir. CAT1is juga digunakan oleh perusahaan telco yang menyediakan layanan ISDN dan PSTN. Dalam kasus seperti ini, pemasangan kabel antara situs pelanggan dan jaringan telco dilakukan dengan menggunakan kabel tipe CAT 1.

CAT2, CAT3, CAT4, CAT5 / 5e, CAT6 & CAT 7 adalah spesifikasi kawat jaringan. Jenis kawat ini bisa mendukung jaringan komputer dan lalu lintas telepon. CAT2 banyak digunakan untuk jaringan token ring, mendukung kecepatan hingga 4 Mbps. Untuk kecepatan jaringan yang lebih tinggi (100 Mbps atau lebih tinggi) CAT5e harus digunakan, namun untuk persyaratan kecepatan 10 Mbps yang hampir punah, CAT3 sudah cukup.

Kabel CAT3, CAT4 dan CAT5 sebenarnya adalah 4 pasang kawat tembaga twisted dan CAT5 memiliki tikungan lebih banyak per inci dari pada CAT3 sehingga dapat berjalan pada kecepatan yang lebih tinggi dan panjang yang lebih besar. Efek "twist" dari masing-masing pasangan di kabel memastikan adanya gangguan yang muncul / diangkat pada satu kabel dibatalkan oleh pasangan kabel yang memutar di sekitar kabel awal. CAT3 dan CAT4 keduanya digunakan untuk jaringan Token Ring - di mana CAT 3 dapat memberikan dukungan maksimal 10Mbps, sementara CAT4 mendorong batas hingga 16Mbps. Kedua kategori tersebut memiliki batas 100 meter.

Kabel CAT5 yang lebih populer kemudian digantikan oleh spesifikasi CAT5e yang memberikan spesifikasi crosstalk yang lebih baik, yang memungkinkannya untuk mendukung kecepatan hingga 1Gbps. CAT5e adalah spesifikasi kabel yang paling banyak digunakan di seluruh dunia dan tidak seperti kabel kategori yang mengikutinya, sangat pemaaf saat panduan penghentian kabel dan penyebaran tidak terpenuhi.

Kabel CAT6 pada awalnya dirancang untuk mendukung gigabit Ethernet, walaupun ada standar yang memungkinkan transmisi gigabit melalui kabel CAT5e. Serupa dengan kabel CAT5e, namun ada pemisah fisik antara keempat pasang untuk mengurangi gangguan elektromagnetik. CAT6 mampu mendukung kecepatan 1Gbps untuk panjang hingga 100 meter, dan 10Gbps juga didukung untuk panjang hingga 55 meter.

Saat ini, sebagian besar instalasi kabel baru menggunakan CAT6 sebagai standar, namun penting untuk dicatat bahwa semua komponen kabel (jack, panel patch, kabel patch dll) harus dilindungi CAT6 dan ekstra hati-hati harus diberikan pada penghentian kabel yang tepat. .

Pada tahun 2009, CAT6A diperkenalkan sebagai kabel spesifikasi yang lebih tinggi, menawarkan imunisasi yang lebih baik pada gangguan crosstalk dan elektromagnetik.

Organisasi yang melakukan instalasi menggunakan kabel CAT6 harus meminta laporan pengujian menyeluruh menggunakan penganalisis kabel bersertifikat, untuk memastikan pemasangan telah dilakukan sesuai dengan pedoman & standar CAT6.

CAT7 adalah spesifikasi kabel tembaga yang lebih baru yang dirancang untuk mendukung kecepatan 10Gbps pada panjang hingga 100 meter. Untuk mencapai hal ini, kabel ini memiliki empat pasang yang terpisah secara perisai plus perisai kabel tambahan untuk melindungi sinyal dari crosstalk dan interferensi elektromagnetik (EMI).

Karena kecepatan data yang sangat tinggi, semua komponen yang digunakan selama pemasangan infrastruktur pengkabelan CAT7 harus disertifikasi CAT7. Ini termasuk panel patch, kabel patch, jack dan konektor RJ-45. Gagal menggunakan komponen bersertifikat CAT7 akan mengakibatkan penurunan kinerja dan kegagalan pengujian CAT7 secara terpisah (misalnya menggunakan Analyzer Kabel) karena standar kinerja CAT7 kemungkinan besar tidak dapat dipenuhi. Saat ini, CAT7 biasanya digunakan di DataCenters untuk koneksi tulang punggung antara server, switch jaringan dan perangkat penyimpanan.






