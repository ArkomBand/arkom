% Nama kelompok : 
% Kelas : D4 TI 1A
% Anggota :
% Muhammad Dzihan Al-Banna	:
% Yusuf Al-Qardhawi			:
% Nurresky					:
% Daffa Naufali Pratama		:







Artikel tentang Storage





\section{Pengertian Storage}
Storage merupakan salah satu perangkat yang digunakan untuk menyimpan hasil dari pemprosesan data. Storage biasanya terdapat didalam komputer,storage ini bisa disebut juga dengan secondary storage.
Storage device dibagi menjadi dua bagian yaitu internal dan eksternal. internal storage device contohnya seperti Hard Disk. Internal Storage ini terdapat dalam komputer. sedangkan Eksternal Storage Device adalah suatu penyimpanan data tambahan pada komputer yang terletak diluar komputer,contohnya Hard Disk Eksternal,Flash Disk,Floppy Disk atau biasa kita sebut disket.
\section{Macam-macam storage Device}

1.Hard Disk Drive

Hard disk merupakan salah satu media penyimpanan data pad komputer yang terdiri dari kumpulan piringan magnetis keras dan berputar,serta komponen elektronik lainnya.Hard disk menggunakan piringan datar yang disebut dengan platter yang pada kedua sisinya dilapisi dengan suatu material yang dirancang agar bisa menyimpan informasi secara magnetis.Platter ini berputar dengan kecepatan tinggi.Setiap permukaan pada platter menampung sati milyar bit data,setiap platter menyimpan informasi dalam lingkaran-lingkaran yang disebut dengan track.Tiap track dipotong-potong lagi menjadi beberapa bagian yang disebut dengan sector.

2.Floppy Disk

Floppy disk drive adalah suatu perangkat penyimpanan yang ada didalam komputer yang dapat menyimpan data dalam kapasitas rendah.Dalam satu komputer bisa terdapat dua floppy sekaligus,tapi biasanya hanya terdapat satu floppy saja yaitu floppy A. Semua jenis floppy dilengkapi dengan unit mekanis seperti driver disk dan head positioner,Drive disk inilah yang membuat disk berputar.selain dapat menyimpan data didalam disket,floppy disk juga dapat untuk boating komputer.
3.Compact Disk

Compact disk ini biasa kita singkat CD adalah sebuah piringan kompak dari jenis piringan optik yang dapat menyimpan data.Compact Disk ini dapat menyimpan data sebesar 700 MB.Untuk membaca CD ini, alat yang diperlukan adalah CD DRIVE.CD ini bersifat hanya dapat dibaca tetapi tidak dapat ditulis,tetapi pada perkembangan terkini CD ini dapat ditulis.

4.Flashdisk

Flashdisk adalah suatu perangkat penyimpanan yang dibuat perangkat dengan minimalis dengan ukuran kecil dengan kapasitas tertentu. Flashdisk ini dibuat dengan
mudah dan simpel karena perangkat ini sangat mudah sekali dipakai dan dibawa kemana saja. Selain itu komponen flashdisk ini mendukung usb 2.0 dan usb 3.0 tergantung
versi base yang dibuat oleh perusahaan flashdisk tersebut. Flashdisk ini mempunyai kapasitas pertama kali diluncurkan dengan ukuran 1 GB dan seiring waktu berjalan
Kapasitas ini semakin diperbesar oleh penemu flashdisk ini hingga 2 tb saat ini. Kecepatan Reading Flashdisk ini berkisar antara 1Mb/s sampai dengan 12Mb/s.

Flashdisk ini dikatakan bahwa flash yang artinya melakukan read and scan, dan disk artinya perangkat storage. Jadi Flashdisk ini bekerja secara Read and Scan untuk
menganalisa isi perangkat tersebut apabila anda menghubungkan sesuai driver usb sesuai dukungan devices. Harga Flashdisk ini dikalangan masyarakat relatif murah
kisaran antara Rp\$ 30ribu sampai dengan Rp\$ 100ribu.

5.Memory Card

Memori Card atau kartu memori adalah sebuah alat yang digunakan untuk menyimpan data.Ukuran memorr cards ini bermacam-macam,mulai dari 126 MB sampai 16 GB.Kartu memori ini ukurannya kecil,tapi dapat menyimpan data dengan ukuran yang besar,terdapat beberapa jenis ukuran memori,tetapi biasanya kartu memori mempunyai ukuran standar bit digital yaitu 16MB,32MB,64MB,128MB,256MB dan seterusnya kelipatan dua.Bukan hanya data dokumen tetapi memori juga bisa menyimpan gambar,video ataupun audio.