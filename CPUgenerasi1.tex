\section{Generasi ke pertama}
	Pada Tahun 1945 IBM memproduksi CPU computer super besar yang dinamakan ENIAC ( Electrical Intregrator and Computer). CPU jenis ini dapat dikatakan sebagai moyangnya computer. ENIAC  terdiri dari 18.000 tabung yang kedap udara. Dalam pengoperasiannya diperlukan ruangan seluas 18x8 meter persegi.
	Pada tahun 1951, CPU generasi pertama mengalami perkembangan dengan lahirnya computer ukuran besar pertama yang bernama EDVAC ( Electronic Discrete Variable Automatic Computer

