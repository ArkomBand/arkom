\section{ASCII}
\subsection{Simbol Dasar}
Untuk perbandingan simbol dasar IPA toWorldbe dengan  Tabel 2 untuk digunakan diacritics dan suprasegmentals. Tabel simbol yang melelahkan diperlihatkan pada Lampiran A. Tersedia satu fonem untuk 12 bahasa di Lampiran B. Saat ini bahasa bahasa yang ada adalah bahasa Denmark, Belanda, Inggris, Prancis, Jerman, Hindi, Jepang, Mandarin Cina, Rusia, Spanyol Castillian, dan Tamil terdaftar.
\subsection{Garis Bawah}
Untuk atopik khusus akan ada diakritik yang terkait dengan simbol utama dengan garis bawah (underscore) dengan lambang \"_"\. Garis bawah (Underscore) tidak akan digunakan untuk tujuan lain. Simbol dasar dengan diakritik umumnya akan digunakan untuk variasi alofonik, dan bukan untuk simbol fonetik dari segmen ucapan biasa di
bahasa dunia ini.
Misalnya, alofon labialized  yang disebabkan karena  di dekatnya
akan dituliskan  dan vokal yang raptical akan menjadi i:r. Simbol diakritik adalah huruf, matematika simbol, angka atau simbol tanda baca. Angka dengan pengecualian 2 dan 0 sebagai diakritik adalah
dicadangkan untuk penetapan nada dalam bahasa nada.
Suprasegmentals ditunjukkan dalam tabel diakritik untuk batas suku kata, primer dan sekunder menekankan. 
Pembuatan untuk intonasi dan aksen pitch juga disediakan. Dalam bahasa nada, nada adalah ditandai dengan diacritics nomor ganjil yang melekat pada vokal pada tingkat fonetis yang luas, karena nada
adalah fonemik Sebenarnya nada itu melekat pada suku kata daripada huruf hidup, tapi ini memungkinkan titik pelekatan label yang konsisten di semua suku kata.
\subsectioon{Fonetik Luas Tambahan di luar IPA}
Beberapa simbol ditambahkan ke set IPA yang dianggap perlu secara akurat
label database pidato kontinu yang telah dikumpulkan selama 5 tahun terakhir. Beberapa di antaranya simbol telah digunakan oleh ahli fonetis di masa lalu, dan beberapa fenomena yang mereka tangkap
telah disebutkan dalam literatur.

