% Nama Kelompok : 
% Kelas : D4 TI 1A
% Anggota : 
% 1. Harun   	1174027
% 2. Fahmi   	1174021
% 3. Kukuh		1174016
% 4. Izzah		1174013
% 5. Rizal		1174014
% 6. Lawimner	1174030





Artikel tentang Bilangan Oktal
\Section{Penjelasan Singkat}
Oktal adalah sebuah sistem bilangan berbasis delapan. Sistem bilangan ini terdiri dari 0,1,2,3,4,5,6,7 dan bilangan ini biasanya dikonversi dari biner yang berkelompok setiap 3 bit. Sistem ini mempersingkat tulisannya, agar menjadi tidak terlalu panjang. Cara baca sistem ini dari ujung paling kanan (LSB kepanjangan dari Least Significant Bit).

Contoh contoh bilangan Oktal,
Misalnya : 
Biner Oktal
000 000 00
000 001 01
000 010 02
000 011 03
Misalnya bilangan oktal 3 adalah hasil dari pengelompokkan dari 000 011, perhitungan secara manual dapat dibuktikan dengan cara seperti berikut ini :
(1 x 21 )+(1 x 20 ) = (1×2)+(1×1) = 3
Dengan menggunakan software ms excel kita dapat melakukan konversi bilangan oktal ke bilangan heksadesimal, bilangan desimal atau pun bilangan biner
 