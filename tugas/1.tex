% Nama Kelompok :
% Kelas :
% 1. dadang NPM
% 2. sarwi NPM
% 3. kus NPM





TUGAS KELOMPOK : 
BUatlah artikel tentang definisi/sejarah/versi/contoh dari :
Kelas 1A
1. Linux
2. MacOS
3. Android
4. Windows
5. FreeBSD/OpenBSD

Kelas 1B
1. Definisi Arsitektur Komputer
2. Software
3. Hardware
4. Kernel
5. Perintah DOS dan UNIX


Artikel diketik menggunakan text editor
Notepad ++
Sublime
Atom
vi / vim

1. file disimpan dalam format namatugas.tex (10)

2. gambar disimpan dalam folder figures dengan namagambar (10)

3. referensi dari google scholar,scholar.google.com (10)

4. Setiap referensi yang diambil, maka tuliskan ke dalam (10)
	file bernama reference.bib
   yang berisi kumpulan bibTex dari referensi nomor 3

5. Gunakan standar pengutipan yang baik dan benar (10)

6. Gambar disebutkan di dalam artikel dengan format \ref{namagambar} (10)
   dan gambar diselipkan dengan menambahkan blok sintaks :
	\begin{figure}[ht]
	\centerline{\includegraphics[width=1\textwidth]{figures/namagambar.JPG}}
	\caption{penjelasan keterangan gambar.}
	\label{namagambar}
	\end{figure}
	Contoh :
	Pada gambar \ref{namagambar} dijelaskan bahwa sistem operasi memiliki 3 versi.
	
7. Referensi disebutkan dengan menyebutkan nama di dalam file bibtex No.4 (10)
   contoh :
	Jika Bibtex :
	@inproceedings{ganapathi2006windows,
	  title={Windows XP Kernel Crash Analysis.},
	  author={Ganapathi, Archana and Ganapathi, Viji and Patterson, David A},
	  booktitle={LISA},
	  volume={6},
	  pages={49--159},
	  year={2006}
	}
	Maka artikel :
	Dalam sebuah artikel dari Ganapathi yang menyebutkan bahwa komputasi 
	adalah keniscayan \cite{ganapathi2006windows}.
	
	
8. Penyebutan subbab dan subsubbab diatur dengan cara (10)
	judul sub bab \section{nama sub bab}
	judul sub sub bab ditulis dengan \subsection{judul sub sub bab}
	judul sub sub sub bab ditulis dengan \subsubsection{Judul sub sub sub bab}
	contoh :
	\section{Sejarah Peta}
	Perkembangan peta dunia tidak luput dari para ahli geografi dan kartografi. Peta dunia yang populer pada saat ini merupkan kontribusi dari para 
	pembuat peta sebelumnya

	\subsection{Ptolemy's}
	Ptolemy's diduga membuat peta pada abad ke 2
	
9. Satu kelompok minimal 1800 kata, waktu maksimal 6*24 jam (10)

10. Harus di scan plagiat setiap kelompok dari portal kampus keren, minimal uniqe 80% (10)

