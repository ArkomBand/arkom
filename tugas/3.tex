tugas
1A
tabel kebenaran
1. Negasi
2. Konjungsi
3. Disjungsi
4. Implikasi
5. Bikondisional


1B 
Hexadecimal dan Binary
1. Konversi Bilangan
2. Operasi Penjumlahan
3. Operasi Pengurangan
4. Operasi Perkalian
5. Operasi Pembagian

1. Tetap Menggunakan Format pada tugas sebelumnya (50)
2. di github satu kelompok membuat grup nama kelompoknya dan masukan semua anggota ke grup tersebut (5)
3. fork buku arkom https://github.com/BukuInformatika/arkom ke grup tersebut (5)
4. commit sehari(min 50 kata) per anggota kelompok selama 6 hari (30)


untuk kode
\begin{verbatim}
a = "anu"
b = "itu"
c = a + b
print(c) 
\end{verbatim}




Format Tambahan :
1. untuk list nomor gunakan
	contoh :
	berikut nama anggota kelompok
\begin{enumerate}
	\item darso
	\item karyo
	\item doyok
\end{enumerate}

2. spesial karakter menggunakan tanda \ didepannya
	contoh :
	\&
	\_
	\"dalam petik\"
	jika spesial karakter menjadi banyak atau satu baris gunakan verb
	contoh :
	\verb|%$'%&$&'%'%'%&'%|
	
3. untuk tabel gunakan table , contoh

\begin{table}[h]
\caption{Small Table}
\centering
\begin{tabular}{ccc}
\hline
one&two&three\\
\hline
C&D&E\\
\hline
\end{tabular}
\end{table}

4. untuk rumus gunakan tag equation
	contoh:
	Rumus bola:

	a) Luas permukaan
	 \begin{equation}
	     L = 4 \pi r^2 \,
	\end{equation}
	b) Volume
	 \begin{equation}
	     V = \frac{4}{3}\pi r^3
	\end{equation}