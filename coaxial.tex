%kelompok 5 Arsitektur Komputer (Kabel Coaxial)
%Tiara Rizki Wulansari (1154026)

\section(Kabel Coaxial)
Di dalam dunia IT khususnya Networking, untuk membentuk suatu jaringan, baik itu bersifat LAN (Local Area Network), maupun WAN (Wide Area Network), kita memerlukan media baik hardware maupun software. Beberapa media hardware yang penting di dalam membangun suatu jaringan adalah kabel atau perangkat Wi-Fi, ethernet card, hub atau switch, repeater, bridge, atau router dll. ada beberapa jenis kabel yang banyak digunakan dan menjadi standart untuk membangun atau sebagai penggunaan komunikasi data dalam jaringan komputer. Namun perlu diingat bahwa hampir 85persen kegagalan yang terjadi pada jaringan komputer disebabkan adanya kesalahan pada media komunikasi yang digunakan termasuk kabel. kabel coaxial salah satunya adalah jenis kabel yang sering digunakan untuk membangun jaringan dalam lingkup LAN (Local Area Network).
dikenal dua jenis tipe kabel coaxial yang digunakan untuk jaringan komputer, yaitu:
\begin{itemize}
	\item * thick coax(mempunyai diameter lumayan besar), dan
	\item * thin coax(mempunyai diameter lebih kecil).
\end{itemize}
