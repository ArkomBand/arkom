\section{Arduino Sensor Gas}
\subsection{Pengertian Arduino}
Arduino adalah perusahaan perangkat keras dan perangkat lunak komputer open-source, proyek, dan komunitas pengguna yang merancang dan memproduksi mikrokontroler board tunggal dan kit mikrokontroler untuk membangun perangkat digital dan objek interaktif yang dapat merasakan dan mengendalikan objek di dunia fisik.
\subsection{sensor gas}
Produk proyek didistribusikan sebagai perangkat keras dan perangkat lunak open-source, yang berlisensi di bawah GNU Lesser General Public License (LGPL) atau GNU General Public License (GPL), yang mengizinkan pembuatan papan Arduino dan distribusi perangkat lunak oleh siapa saja. Papan Arduino tersedia secara komersil dalam bentuk preassembled, atau sebagai kit do-it-yourself (DIY)\cite{kushner2011making}


\begin{verbatim}
int redLed = 12;
int greenLed = 11;
int buzzer = 10;
int smokeA0 = A5;
// Your threshold value
int sensorThres = 400;

void setup() {
  pinMode(redLed, OUTPUT);
  pinMode(greenLed, OUTPUT);
  pinMode(buzzer, OUTPUT);
   pinMode(smokeA0, INPUT);
  Serial.begin(9600);
}

void loop() {
  int analogSensor = analogRead(smokeA0);

  Serial.print("Pin A0: ");
  Serial.println(analogSensor);
  // Checks if it has reached the threshold value
  if (analogSensor > sensorThres)
  {
    digitalWrite(redLed, HIGH);
    digitalWrite(greenLed, LOW);
    digitalWrite(buzzer, HIGH);
  }
  else
  {
    digitalWrite(redLed, LOW);
    digitalWrite(greenLed, HIGH);
    digitalWrite(buzzer, LOW);
  }
  delay(100);
}
\end{verbatim}