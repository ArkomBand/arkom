\section{Arduino}
\subsection{Pengertian Arduino}
Arduino merupakan sebuah perangkat mikro dari Singleboard yang bersifat Opensource yang berasal dari platform Wiring
yang kemudian dirancang untuk dapat memudahkan para penggunaan elektronik di setiap bidang. Perangkat kerasnya memiliki prosesor AVR Atmel, 
nama AVR sendiri berasal dari nama prosesor Alf (Egil Bogen) dan Risc Vegard (Wollan) "di mana Alf Egil Bogen dan Vegard Wollan adalah dua 
penemu yang berasal dari Norwegia yang menemukan mikrokontroler AVR yang kemudian dipasarkan oleh Atmel, dan perangkat lunaknya tersebut memiliki bahasa pemrograman tersendiri. 

Arduino juga merupakan platform perangkat keras terbuka yang ditujukan untuk siapa pun dan kalangan apapun yang ingin membuat prototip 
peralatan elektronik interaktif berdasarkan perangkat keras dan perangkat lunak yang fleksibel dan mudah digunakan. Perangkat dari
mikrokontroler deprogram atau dibuat menggunakan bahasa pemrograman arduino yang memiliki kesamaan atau kemiripan dengan bahasa
pemrograman C, karena bersifat terbuka dapat mendownload skema hardware arduino dan membangunnya. 

\subsection{jenis-jenis Arduino}
Arduino Uno R3
Arduino Uno R3 adalah boardsistem minimum berbasis mikrokontroller ATmega328P jenis AVR. Arduino Uno R3 memiliki 14 digital input/output (6
diantaranya dapat digunakan untuk PWM output), 6 analog input, 16 MHz osilator kristal, USB connection, power jack, ICSP header dan tombol reset. Skema dari Arduino Uno R3 dengan
karekteristik sebagai berikut:
• Operating voltage 5 VDC.
• Rekomendasi input voltage 7-12
VDC
• Batas input voltage 6-20 VDC.
• Memiliki 14 buah input/output
digital.
• Memiliki 6 buah input analog.
• DC Current setiap I/O Pin sebesar
40mA.
• DC Current untuk 3.3V Pin sebesar
50mA.
• Flash memory 32 KB.
• SRAM sebesar 2 KB.
• EEPROM sebesar 1 KB.
• 11 Clock Speed 16 MHz.

\subsubsection{Arduino Mega}

Arduino Mega adalah papan mikrokontroler berdasarkan ATmega1280 (datasheet). Ini memiliki 54 pin input atau output digital, 16 input analog, 4 UART (port serial perangkat keras), osilator kristal 16 MHz, koneksi USB, colokan listrik, header ICSP, dan tombol reset. Ini berisi semua yang dibutuhkan untuk mendukung mikrokontroler; cukup hubungkan ke komputer dengan kabel USB atau nyalakan dengan adaptor AC-ke-DC atau baterai untuk memulai. Mega kompatibel dengan kebanyakan perisai yang dirancang untuk Arduino Duemilanove atau Diecimila.
