\section{Arduino}
\subsection{Pengertian Arduino}
Arduino merupakan sebuah perangkat mikro dari Singleboard yang bersifat Opensource yang berasal dari platform Wiring yang kemudian dirancang untuk dapat memudahkan para penggunaan elektronik di setiap bidang. Perangkat kerasnya memiliki prosesor AVR Atmel, nama AVR sendiri berasal dari nama prosesor Alf (Egil Bogen) dan Risc Vegard (Wollan) "di mana Alf Egil Bogen dan Vegard Wollan adalah dua penemu yang berasal dari Norwegia yang menemukan mikrokontroler AVR yang kemudian dipasarkan oleh Atmel, dan perangkat lunaknya tersebut memiliki bahasa pemrograman tersendiri. 

Arduino juga merupakan platform perangkat keras terbuka yang ditujukan untuk siapa pun dan kalangan apapun yang ingin membuat prototip peralatan elektronik interaktif berdasarkan perangkat keras dan perangkat lunak yang fleksibel dan mudah digunakan. Perangkat dari mikrokontroler deprogram atau dibuat menggunakan bahasa pemrograman arduino yang memiliki kesamaan atau kemiripan dengan bahasa pemrograman C, karena bersifat terbuka dapat mendownload skema hardware arduino dan membangunnya. 

\subsection{jenis-jenis Arduino}
Arduino Uno R3
Arduino Uno R3 adalah boardsistem minimum berbasis mikrokontroller ATmega328P jenis AVR. Arduino Uno R3 memiliki 14 digital input/output (6
diantaranya dapat digunakan untuk PWM output), 6 analog input, 16 MHz osilator kristal, USB connection, power jack, ICSP header dan tombol reset. Skema dari Arduino Uno R3 dengan
karekteristik sebagai berikut:
• Operating voltage 5 VDC.
• Rekomendasi input voltage 7-12
VDC
• Batas input voltage 6-20 VDC.
• Memiliki 14 buah input/output
digital.
• Memiliki 6 buah input analog.
• DC Current setiap I/O Pin sebesar
40mA.
• DC Current untuk 3.3V Pin sebesar
50mA.
• Flash memory 32 KB.
• SRAM sebesar 2 KB.
• EEPROM sebesar 1 KB.
• 11 Clock Speed 16 MHz.

\subsection{Kerja Arduino}
Dengan beberapa dasar-dasar listrik, Arduino, dan robot.
contoh kode menggunakan komponen berdaya rendah dapat dihubungkan langsung ke Arduino (LED, potensiometer, penerima R / C, tombol switch, dan sebagainya). 
Bab ini berfokus pada bagaimana menghubungkan Arduino dengan switch mekanis, elektronik, dan optik, serta beberapa metode kontrol masukan yang berbeda, dan akhirnya beberapa pembicaraan tentang sensor.

\subsubsection{Arduino Mega}

Arduino Mega adalah papan mikrokontroler berdasarkan ATmega1280 (datasheet). Ini memiliki 54 pin input atau output digital, 16 input analog, 4 UART (port serial perangkat keras), osilator kristal 16 MHz, koneksi USB, colokan listrik, header ICSP, dan tombol reset. Ini berisi semua yang dibutuhkan untuk mendukung mikrokontroler; cukup hubungkan ke komputer dengan kabel USB atau nyalakan dengan adaptor AC-ke-DC atau baterai untuk memulai. Mega kompatibel dengan kebanyakan perisai yang dirancang untuk Arduino Duemilanove atau Diecimila.

\subsubsection{Sensor DHT 11}
Sensor ini merupakan sensor dengan kalibrasi sinyal digital yang mampu memberikan informasi suhu dan kelembaban. Sensor ini tergolong komponen yang memiliki tingkat stabilitas yang sangat baik. Sensor ini termasuk elemen resistif dan perangkat pengukur suhu NTC.
Memiliki kualitas yang sangat baik, respon cepat, dan dengan harga yang
terjangkau. DHT11 memiliki fitur kalibrasi yang sangat akurat. Koefisien kalibrasi ini disimpan dalam OTP program memory,sehingga ketika internal sensor mendeteksi sesuatu, maka module ini membaca koefisien sensor tersebut. produk ini cocok digunakan untuk banyak aplikasi-aplikasi pengukuran suhu dan kelembaban.

\subsubsection{Perekam Data (Data Logger)}
 Perekam Data disebut juga data logger. Secara umum perekam data sederhana terdiri dari mikrokontroller,sensor dan media penyimpanan.
Mikrokontroller merupakan bagian dari perekam data yang mengatur komunikasi antar perangkat. Sensor berfungsi untuk mengubah sinyal analog manjadi sinyal digital. Media penyimpanan berfungsi untuk menyimpan data Dalam sistem telemetri ini terdapat fitur data logger,yaitu fitur yang berfungsi sebagai penyimpanan semua data-data kondisi dari suhu dan kelembaban yang diukur. Kemudian Data ini nantinya akan
tersimpan didalam media penyimpanan yaitu memory card. Pada perancangan ini jenis memory card yang akan digunakan adalah micro SD 
(Secure Digital) dengan kapasitas 4 GB.

\subsubsection{Topologi Jaringan}
Sistem pemantauan dan pengukuran jarak jauh terdiri dari 2 buah modul 
Xbee Proyang sama yang sebelumnya telah diprogram sebagai sebuah receiver-transmiter maupun transmiter-receiver. Ada beberapa bentuk topologi yang biasa digunakan antara lain topologi mesh, peer, star, dan cluster Tree.
Topologi pair merupakan jaringan yang sederhana dengan hanya menggunakan dua buah xbeeatau node. Satu node harus menjadi coordinator sehingga jaringan dapat dibentuk. Dan yang lain dikonfigurasikan sebagai routeratau perangkat akhir.
