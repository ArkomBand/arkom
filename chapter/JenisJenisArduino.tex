\section{Arduino}
\subsection{Pengertian Arduino}
Arduino merupakan sebuah perangkat mikro dari Singleboard yang bersifat Opensource yang berasal dari platform Wiring yang kemudian dirancang untuk dapat memudahkan para penggunaan elektronik di setiap bidang. Perangkat kerasnya memiliki prosesor AVR Atmel, nama AVR sendiri berasal dari nama prosesor Alf (Egil Bogen) dan Risc Vegard (Wollan) "di mana Alf Egil Bogen dan Vegard Wollan adalah dua penemu yang berasal dari Norwegia yang menemukan mikrokontroler AVR yang kemudian dipasarkan oleh Atmel, dan perangkat lunaknya tersebut memiliki bahasa pemrograman tersendiri. 
Arduino juga merupakan platform perangkat keras terbuka yang ditujukan untuk siapa pun dan kalangan apapun yang ingin membuat prototip peralatan elektronik interaktif berdasarkan perangkat keras dan perangkat lunak yang fleksibel dan mudah digunakan. Perangkat dari mikrokontroler deprogram atau dibuat menggunakan bahasa pemrograman arduino yang memiliki kesamaan atau kemiripan dengan bahasa pemrograman C, karena bersifat terbuka dapat mendownload skema hardware arduino dan membangunnya. 
Arduino menggunakan keluarga mikrokontroler ATMega yang dilepaskan Atmel sebagai basis, namun ada individu / perusahaan yang membuat klon arduino menggunakan mikrokontroler lainnya dan tetap kompatibel dengan arduino di tingkat perangkat keras. Agar bisa, program dimuatkan melalui bootloader meski ada pilihan untuk bypass bootloader dan menggunakan downloader untuk memprogram mikrokontroler secara langsung melalui port ISP.

\subsection{Sejarah Arduino}
Semuanya dimulai dengan tesis yang dibuat oleh Hernando Barragan, di institut Ivrea, Italia pada tahun 2005, dikembangkan oleh Massimo Banzi dan David Cuartielles dan dinamai Arduin dari Ivrea. Kemudian berganti nama menjadi Arduino yang di Italia berarti teman yang pemberani.
Tujuan awal Arduino adalah membuat perangkat mudah dan murah, dari perangkat yang ada saat itu. Dan perangkat ini diperuntukkan bagi siswa yang akan membuat perangkat desain dan interaksi.
Saat ini tim pengembangnya adalah Massimo Banzi, David Cuartielles, Tom Igoe, Gianluca Martino, David Mellis, dan Nicholas Zambetti. Mereka mencoba empat hal dalam bahasa Arduino ini:
1. Harga terjangkau
2. Bisa dijalankan di berbagai sistem operasi, Windows, Linux, Max, dan sebagainya.
3. Sederhana, dengan bahasa pemrograman yang mudah dipelajari oleh orang awam, bukan untuk orang teknis saja.
4. Open Source, perangkat keras dan perangkat lunak.


\subsection{jenis-jenis Arduino}
Arduino Uno R3
Arduino Uno R3 adalah boardsistem minimum berbasis mikrokontroller ATmega328P jenis AVR. Arduino Uno R3 memiliki 14 digital input/output (6
diantaranya dapat digunakan untuk PWM output), 6 analog input, 16 MHz osilator kristal, USB connection, power jack, ICSP header dan tombol reset. Skema dari Arduino Uno R3 dengan
karekteristik sebagai berikut:
• Operating voltage 5 VDC.
• Rekomendasi input voltage 7-12
VDC
• Batas input voltage 6-20 VDC.
• Memiliki 14 buah input/output
digital.
• Memiliki 6 buah input analog.
• DC Current setiap I/O Pin sebesar
40mA.
• DC Current untuk 3.3V Pin sebesar
50mA.
• Flash memory 32 KB.
• SRAM sebesar 2 KB.
• EEPROM sebesar 1 KB.
• 11 Clock Speed 16 MHz.

\subsection{Kerja Arduino}
Dengan beberapa dasar-dasar listrik, Arduino, dan robot.
contoh kode menggunakan komponen berdaya rendah dapat dihubungkan langsung ke Arduino (LED, potensiometer, penerima R / C, tombol switch, dan sebagainya). 
Bab ini berfokus pada bagaimana menghubungkan Arduino dengan switch mekanis, elektronik, dan optik, serta beberapa metode kontrol masukan yang berbeda, dan akhirnya beberapa pembicaraan tentang sensor.

\subsection{Anatomi Jaringan Sensor}
Jaringan sensor ada dimana-mana. Mereka biasanya dianggap sebagai sistem pemantauan manufaktur yang rumit
dan aplikasi medis. Namun, mereka tidak selalu rumit, dan mereka ada di sekitar Anda.
Di bagian ini. kita akan memeriksa blok bangunan dari jaringan sensor. dan bagaimana mereka terhubung (secara logis).
pertama mari kita lihat contoh dari jaringan sensor dalam upaya memvisualisasikan komponennya.

\subsection{Mengapa menggunakan Arduino?}
Fleksibel, menawarkan beragam input digital dan analog, SPI dan interface serial serta digital dan PWM
output. Mudah digunakan, terhubung ke komputer via USB dan berkomunikasi menggunakan protokol serial standar, berjalan
dalam mode standalone dan sebagai antarmuka yang terhubung ke komputer PC / Macintosh
Ini tidak mahal, sekitar 30 dollar per papan dan dilengkapi dengan perangkat lunak authoring gratis. Ini adalah proyek sumber terbuka,
Perangkat lunak / perangkat keras sangat mudah diakses dan sangat fleksibel untuk disesuaikan dan diperluas. Arduino didukung
oleh komunitas online yang berkembang, banyak sumber sudah tersedia.

\subsubsection{Arduino berinteraksi dengan Softwares}
Arduino dapat \"berbicara\", (mentransmisikan atau menerima data) melalui saluran serial, jadi perangkat lain dengan serial
Kemampuan bisa berkomunikasi dengan Arduino. Tidak masalah bahasa program / pemrograman apa yang sedang mengemudi
perangkat lainnya Anda bisa menggunakan port serial \"utama\" Arduino, yang digunakan saat Anda \"berbicara\" dengannya
program itu, atau Anda dapat meninggalkan saluran yang didedikasikan untuk pemrograman (dan serial lingkungan pengembangan
monitor), dan gunakan dua pin lainnya untuk link serial tambahan yang didedikasikan untuk perangkat eksternal. Beberapa program (seperti
Flash) tidak memiliki kemampuan serial asli. Mereka masih bisa berkomunikasi dengan Arduino melalui perantara
yang, seperti \"penerjemah\", memungkinkan mereka berbicara satu sama lain.

\subsubsection{Arduino Mega}

Arduino Mega adalah papan mikrokontroler berdasarkan ATmega1280 (datasheet). Ini memiliki 54 pin input atau output digital, 16 input analog, 4 UART (port serial perangkat keras), osilator kristal 16 MHz, koneksi USB, colokan listrik, header ICSP, dan tombol reset. Ini berisi semua yang dibutuhkan untuk mendukung mikrokontroler; cukup hubungkan ke komputer dengan kabel USB atau nyalakan dengan adaptor AC-ke-DC atau baterai untuk memulai. Mega kompatibel dengan kebanyakan perisai yang dirancang untuk Arduino Duemilanove atau Diecimila.

\subsubsubsection{Power}
Arduino Mega dapat bertenaga melalui koneksi USB atau dengan catu daya eksternal. Sumber daya dipilih secara otomatis.

Daya eksternal (non-USB) bisa datang baik dari adaptor AC-ke-DC (kutil dinding) atau baterai. Adaptor dapat dihubungkan dengan memasang steker positif pusat 2.1mm ke soket daya board. Memimpin dari baterai dapat dimasukkan ke dalam header pin Gnd dan Vin pada konektor POWER.

Papan dapat beroperasi pada suplai eksternal 6 sampai 20 volt. Jika dipasok dengan kurang dari 7V, pin 5V dapat memasok kurang dari lima volt dan board mungkin tidak stabil. Jika menggunakan lebih dari 12V, regulator tegangan mungkin terlalu panas dan merusak board. Kisaran yang disarankan adalah 7 sampai 12 volt.

Pin Power adalah sebagai berikut:
\begin{itemize}
\item VIN. Tegangan masukan ke papan Arduino saat menggunakan sumber daya eksternal (berlawanan dengan 5 volt dari koneksi USB atau sumber daya yang diatur lainnya). Anda bisa mensuplai voltase melalui pin ini, atau jika mensuplai voltase melalui colokan listrik, akseslah melalui pin ini.
\item 5V. Catu daya yang diatur digunakan untuk menyalakan mikrokontroler dan komponen lainnya di papan tulis. Ini bisa datang baik dari VIN melalui regulator onboard, atau disediakan oleh USB atau suplai 5V yang diatur lainnya.
\item 3V3 Pasokan 3,3 volt dihasilkan oleh chip FTDI onboard. Hasil maksimum saat ini adalah 50 mA.
\item GND. Ground pin
\end{itemize}

\subsubsubsection{communication}

Arduino Mega memiliki sejumlah fasilitas untuk berkomunikasi dengan komputer, Arduino lain, atau mikrokontroler lainnya. ATmega1280 menyediakan empat perangkat keras UART untuk komunikasi serial TTL (5V). FTDI FT232RL di papan saluran salah satu dari ini melalui USB dan driver FTDI (disertakan dengan perangkat lunak Arduino) menyediakan port com virtual untuk perangkat lunak pada komputer. Perangkat lunak Arduino mencakup monitor serial yang memungkinkan data tekstual sederhana dikirim ke dan dari dewan Arduino. LED RX dan TX di papan akan berkedip saat data dikirimkan melalui chip FTDI dan koneksi USB ke komputer (tapi tidak untuk komunikasi serial pada pin 0 dan 1).

Sebuah perangkat lunak Serial library memungkinkan komunikasi serial pada salah satu pin digital Mega.

ATmega1280 juga mendukung komunikasi I2C (TWI) dan SPI. Perangkat lunak Arduino mencakup perpustakaan Wire untuk menyederhanakan penggunaan bus I2C; lihat dokumentasi di situs Wiring untuk rinciannya. Untuk menggunakan komunikasi SPI, silakan lihat datasheet ATmega1280.

\subsubsubsection{Karakteristik Fisik dan Kompatibilitas}
Panjang maksimum dan lebar PCB Mega masing-masing adalah 4 dan 2,1 inci, dengan konektor USB dan colokan listrik melampaui dimensi sebelumnya. Tiga lubang sekrup memungkinkan papan dipasang pada permukaan atau kotak. Perhatikan bahwa jarak antara pin 7 dan 8 digital adalah 160 mil (0,16 "), bukan kelipatan jarak 100 mil dari pin lainnya.

Mega dirancang agar kompatibel dengan sebagian besar perisai yang dirancang untuk Diecimila atau Duemilanove. Pin digital 0 sampai 13 (pin AREF dan GND yang berdekatan), input analog 0 sampai 5, soket daya, dan header ICSP semuanya berada pada lokasi yang setara. Selanjutnya, UART utama (port serial) terletak pada pin yang sama (0 dan 1), seperti halnya interupsi eksternal 0 dan 1 (pin 2 dan 3 masing-masing). SPI tersedia melalui header ICSP pada Mega dan Duemilanove / Diecimila. Harap dicatat bahwa I2C tidak terletak pada pin yang sama pada Mega (20 dan 21) sebagai Duemilanove / Diecimila (input analog 4 dan 5).

\subsubsubsection{Perlindungan Overcurrent USB}
Arduino Mega memiliki polibak yang dapat disetel ulang yang melindungi port USB komputer Anda dari celana pendek dan arus lebih. Meskipun kebanyakan komputer menyediakan perlindungan internal mereka sendiri, sekering menyediakan lapisan perlindungan ekstra. Jika lebih dari 500 mA diterapkan ke port USB, sekering akan secara otomatis memutus koneksi sampai pendek atau overload dilepaskan.
\subsubsection{Sensor DHT 11}
Sensor ini merupakan sensor dengan kalibrasi sinyal digital yang mampu memberikan informasi suhu dan kelembaban. Sensor ini tergolong komponen yang memiliki tingkat stabilitas yang sangat baik. Sensor ini termasuk elemen resistif dan perangkat pengukur suhu NTC.
Memiliki kualitas yang sangat baik, respon cepat, dan dengan harga yang
terjangkau. DHT11 memiliki fitur kalibrasi yang sangat akurat. Koefisien kalibrasi ini disimpan dalam OTP program memory,sehingga ketika internal sensor mendeteksi sesuatu, maka module ini membaca koefisien sensor tersebut. produk ini cocok digunakan untuk banyak aplikasi-aplikasi pengukuran suhu dan kelembaban.

\subsubsection{Perekam Data (Data Logger)}
 Perekam Data disebut juga data logger. Secara umum perekam data sederhana terdiri dari mikrokontroller,sensor dan media penyimpanan.
Mikrokontroller merupakan bagian dari perekam data yang mengatur komunikasi antar perangkat. Sensor berfungsi untuk mengubah sinyal analog manjadi sinyal digital. Media penyimpanan berfungsi untuk menyimpan data Dalam sistem telemetri ini terdapat fitur data logger,yaitu fitur yang berfungsi sebagai penyimpanan semua data-data kondisi dari suhu dan kelembaban yang diukur. Kemudian Data ini nantinya akan
tersimpan didalam media penyimpanan yaitu memory card. Pada perancangan ini jenis memory card yang akan digunakan adalah micro SD 
(Secure Digital) dengan kapasitas 4 GB.

\subsubsection{Topologi Jaringan}
Sistem pemantauan dan pengukuran jarak jauh terdiri dari 2 buah modul 
Xbee Proyang sama yang sebelumnya telah diprogram sebagai sebuah receiver-transmiter maupun transmiter-receiver. Ada beberapa bentuk topologi yang biasa digunakan antara lain topologi mesh, peer, star, dan cluster Tree.
Topologi pair merupakan jaringan yang sederhana dengan hanya menggunakan dua buah xbeeatau node. Satu node harus menjadi coordinator sehingga jaringan dapat dibentuk. Dan yang lain dikonfigurasikan sebagai routeratau perangkat akhir.

Cara yang saat ini banyak digunakan  dalam topologi jaringan adalah bus, token ring, dan star yaitu:
1. Topologi BUS 
Topologi bus terlihat pada Gambar 2. Media penghantar untuk jenis topologi BUS adalah kabel Koaksial. Topologi BUS menggunakan metode unicast, multicastdan broadcast. Unicastadalah komu- nikasi antara satu pengirim 
dengan satu penerima di jaringan. Multicastadalah komunikasi antara satu pengirim dengan banyak penerima di jaringan. Sedangkan pada Broadcast, setiap titik akan menerima dan menyimpan frameyang disalurkan/dihantarkan.

2. Topologi Token RING 
Topologi Token RING adalah. Metode token-ring(sering disebut ringsaja) menghu-bungkan komputer sehingga ber-bentuk ring (lingkaran). Setiap 
simpul mempunyai tingkatan yang sama. Jaringan akan disebut sebagai loop, data dikirimkan kesetiap simpul dan setiap informasi yang diterima simpul diperiksa alamatnya apakah data itu untuknya atau bukan.

3. Topologi STAR 
Topologi ini merupakan kontrol terpusat, semua link harus melewati pusat yang menyalurkan data tersebut kesemua simpul atau clientyang dipilihnya. Simpul pusat dinamakan stasiun primer atau 
server dan lainnya dinamakan 
stasiun sekunder atau client server.Setelah hubungan jaringan dimulai oleh servermaka setiap client serversewaktu-waktu dapat menggu-nakan hubungan jaringan ter-sebut tanpa menunggu perintah dari server.

\subsubsection {Arduino 1.0.1 dan XCTU}
Arduino merupakan perangkatpemrograman mikrokontroller jenis Atmel yang tersedia secara bebas (open-source)
dengan menggunakan bahasapemrograman C. Untuk menyelesaikan rangkaian agar bisa bekerja, maka langkah selanjutnya adalah membuat program yang
akan diupload ke board Arduino.Penelitian ini menggunakan software Arduino 1.0.1untuk membuat program pada sketch Arduino kemudian diverify
untuk memastikan program sudah benar,selanjutnya program di upload. Setelah program di upload dan tidak ada kesalahan
maka akan tampil done uploading. Untuk mengaplikasikan program pada sistem telemetri ini maka dibutuhkan perangkat lunak yang untuk men-setting atau pun pemberian alamat pada Xbee Pro untuk melakukan komunikasi antara unit pengirim dan unit penerima. Adapun
perangkat lunak yang di gunakan adalah perangkat lunak X-CTU, yaitu perangkat lunak dari produk Xbee Pro

\subsubsection  {I/O Expansion Shield Arduino}
I/O Expansion Shield untuk Arduino adalah perangkat tambahan yang digunakan untuk interface beberapa modul yang compatible dengan board arduino. Board I/O expansion ini memiliki inputtegangan 5 VDC. Modul- modul yang cocok dan sesuai dengan board Arduino dapat mendukung RS485. Xbee Pro, APC220, SD Card dan Bloetooth.

\subsubsection {Xbee Pro}
 Xbee Pro merupakan modul yang memungkinkan Arduino Uno untuk berkomunikasi secara wireless
mengunakan protocol ZigBee. ZigBee beroperasi menggunakan pada spesifikasi IEEE 802.15.4 beroperasi pada frekuensi
2.4 GHz, 900 dan 868 MHz. XBee Pro dapat digunakan sebagai pengganti kabel serial.Xbee Pro diharapkan dapat memperkecil biaya dan menjadi
konektivitas berdaya rendah untuk peralatan yang memerlukan baterai untuk hidup selama beberapa bulan sampai beberapa tahun, tetapi tidak memerlukan kecepatan transfer data tinggi. Xbee Pro memungkinkan komunikasi wireless dalam jangkauan hingga 100 meter indoor dan 1500 meter outdoor.


\subsubsection {JENIS JENIS ARDUINO USB}
Menggunakan USB sebagai antar muka pemrograman atau komunikasi komputer. Contoh:
Arduino Uno
Arduino Duemilanove
Arduino Diecimila
Arduino NG Rev. C
Arduino NG (Nuova Generazione)
Arduino Extreme dan Arduino Extreme v2
Arduino USB dan Arduino USB v2.0 

\subsubsection {ARDUINO NANO DAN ARDUINO MINI}
Papan berbentuk kompak dan digunakan bersama breadboard. Contoh seperti:
Arduino Nano 3.0, Arduino Nano 2.x
Arduino Mini 04, Arduino Mini 03, Arduino Stamp 02