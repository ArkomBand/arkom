PENGERTIAN BILANGAN BINER ATAU BINARY

Bilangan biner atau bisa juga disebut bilangan binary merupakan sistem penulisan angka dengan hanya menggunkan dua simbol yaitu 1 dan 2. bilangan biner merupakan dasardari semua sistem bilangan yang berbasis digital. dari sistem biner kita dapat mengkonversikannya ke sistem bilangan Oktal atau Hexadesimal.

Bilangan biner umumnya digunakan dalam dunia komputasi. komputer menggunakan bilangan biner agar dapat saling berinteraksi terhadap semua komponen (hardware) dan bisa juga berinteraksi terhadap sesama komputer. contoh nya pada sebuah komputer yaitu apabila sebuah komputer terhubung dengan tegangan listrik maka bernilai 1 dan apabila komputer tidak terhubung dengan jaringan listrik makanilai nya 0.
operasi bilangan biner  adalah operasi antara dua bilangan. dasar perkalian adalah tabel yang memuat hasil perkalian operasi pada biner antara bilangan satu digit.
