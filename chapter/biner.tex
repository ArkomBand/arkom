% Nama Kelompok : Kelompok 2
% Kelas : D4 TI 1A
% 1. Kadek Diva Krishna Murti (1174006)
% 2. Duvan Silalahi (1174011)
% 3. Oniwaldus (1174005)
% 4. Choirul Anam (1174004)
% 5. Sri Rahayu (1174015)
% 6. Ilham Habibi (1174028)

\documentclass{article}

\begin{document}

\section{Pengertian} 


Sejak pertama kali komputer elektronik digunakan, komputer tersebut telah beroperasi dengan menggunakan sistem bilangan biner, yaitu bilangan berbasis dua pada sistem bilangan. Semua kode program dan data pada komputer disimpan serta dimanipulasi dalam format biner yang merupakan kode - kode mesin komputer. Sehingga semua perhitungan – perhitungan yang diolah oleh computer tersebut menggunakan aritmatika biner yang hasilnya berupa bilangan hanya memiliki dua kemungkinan nilai, yaitu 0 dan 1. 

Bilangan biner atau bilangan berbasis dua atau binary dalam Bahasa Inggris merupakan sebuah penulisan bilangan di mana bilangan – bilangan tersebut hanya menggunakan dua angka, yaitu 0 dan 1. Tidak seperti bilangan desimal yang merupakan sistem bilangan berbasis 10, sistem bilangan biner berbasis 2. bilangan biner digunakan untuk informasi biner dan juga satuan ukuran besarnya data. Sistem bilangan biner modern detemukan oleh Gottfried Wilhelm Leibniz pada abad ke-17. Sistem bilangan ini merupakan dasar dari semua sistem bilangan berbasis digital. Dari sistem biner , kita dapat mengkonversinya ke sistem bilangan Oktal atau Hexadesimal. Sistem ini juga dapat kita sebut dengan istilah bit atau Binary Digit atau dalam arsitektur elektronik biasa disebut sebagai digital logic.. 

Pengelompokan biner dalam komputer selalu berjumlah 8, dengan istilah 1 Byte atau bita. Dalam istilah komputer, 1 Byte = 8 bit. Kode-kode rancang bangun sistem bilangan berbasis 10, sistem bilangan biner berbasis 2. Bilangan biner digunakan untuk informasi biner dan juga satuan ukuran besarnya data.

Sistem bilangan biner modern ditemukan oleh Gottfried Wilhelm Leibniz pada abad ke-17. Sistem bilangan ini merupakan dasar dari semua sistem bilangan berbasis digital. Dari sistem biner, kita dapat mengkonversinya ke sistem bilangan Oktal atau Hexadesimal. Sistem ini juga dapat kita sebut dengan istilah bit atau Binary Digit atau dalam arsitektur elektronik biasa disebut sebagai digital logic. Pengelompokan biner dalam komputer selalu berjumlah 8, dengan istilah 1 Byte atau bita. Dalam istilah komputer, 1 Byte = 8 bit. Kode-kode rancang bangun komputer seperti ASCII, American Standard Code for Information Interchange menggunakan sistem pengkodean 1 Byte.

Setiap digit pada bilangan biner mewakili pangkat pada angka 2 yang terus meningkat dari kanan ke kiri, Digit yang paling kanan mewakili 2 pangkat 0 ($2^0$). digit selanjutnya mewakili 2 pangkat 1 ($2^1$), selanjutnya lagi mewakili mewakili 2 pangkat 2 ($2^2$), dan seterusnya. Bilangan desimal 0 diwakili dengan bilangan biner '0', begitu juga dengan bilangan desimal 1 diwakili dengan bilangan biner '1'. Kedua bilangan 1 dan 0 tersebut tidak berubah. Bilangan desimal 2 diwakili sebagai bilangan biner '10', 3 sebagai '11', 4 sebagai '100', 5 sebagai '101', dan seterusnya.

Dalam sistem komunikasi digital modern, dimana data ditransmisikan dalam bentuk bit-bit biner, dibutuhkan sistem yang tahan terhadap noise yang terdapat di kanal transmisi sehingga data yang ditransmisikan tersebut dapat diterima dengan benar. Kesalahan dalam pengiriman atau penerimaan data merupakan permasalahan yang mendasar yang memberikan dampak yang sangat signifikan pada sistem komunikasi.Biner yang biasa dipakai itu ada 8 digit angka dan cuma berisikan angka 1 dan 0, tidak ada angka lainnya.
Posisi sebuah angka dalam bilangn biner atau bilangan basis dua akan menentukan berapa bobot nilainya. Posisi paling depan (kiri) sebuah bilangan memiliki nilai yang paling besar sehingga disebut sebagai MSB (Most Significant Bit), dan posisi paling belakang (kanan) sebuah bilangan memiliki nilai yang paling kecil sehingga disebut sebagai LSB (Leased Significant Bit). Berikut ini adalah contoh representasi dari bilangan biner atau bilangan berbasis dua : 
$10110_2$ = 1 x $2^4$ + 0 x $2^3$ + 1 x $2^2$ + 1 x $2^1$ + 0 x $2^0$ = $22_{10}$

\subsection{Bilangan Biner}
Sebagai contoh dari bilangan desimal, untuk angka 157 : $157_{(10)}$ = (1 x 100) + (5 x 10) + (7 x 1) 
Perhatikan! Bilangan desimal ini sering juga disebut dengan basis 10. Hal ini dikarenakan perpangkatan 10 yang didapat dari 100, 101, 102, dst. 
\subsubsection{Mengenal Konsep Bilangan Biner dan Desimal}
Perbedaan paling mendasar dari metode bilangan biner dan bilangan desimal terletak pada jumlah dari basisnya. Jika desimal berbasis 10(X10) berpangkatkan 10x, maka untuk bilangan biner berbasiskan 2 (x2) menggunakan perpangkatan 2x. Sederhananya perhatikan contoh dibawah ini!
$14_{(10)}$=(1x$^1$) + (4 x $10^0$)
=10 + 4
=14

Bentuk umum dari bilangan biner dan bilangan desimal bisa dilihat pada tabel \ref{table:binerdesimal}. 

\begin{table}[h!]
\centering
\begin{tabular}{ |c|c|c|c|c|c|c|c|c|c| } 
\hline
Biner & 1 & 1 & 1 & 1 & 1 & 1 & 1 & 1 & 11111111 \\ 
\hline
Desimal & 128 & 64 & 32 & 16 & 8 & 4 & 2 & 1 & 255 \\ 
\hline
Pangkat & $2^7$ & $2^6$ & $2^5$ & $2^4$ & $2^3$ & $2^2$ & $2^1$ & $2^0$ & $X^{1-7}$ \\ 
\hline
\end{tabular}
\caption{Tabel bentuk umum dari bilangan biner dan bilangan desimal}
\label{table:binerdesimal}
\end{table}
Sekarang kita balik lagi ke contoh soal di atas! Darimana kita dapatkan angka desimal 14(10) menjadi angka biner 1110(2)? 
Mari kita lihat lagi pada bentuk umumnya pada tabel \ref{table:contoh1}!

\begin{table}[h!]
\centering
\begin{tabular}{ |c|c|c|c|c|c|c|c|c|c| } 
\hline
Biner & 0 & 0 & 0 & 0 & 1 & 1 & 1 & 0 & 00001110 \\ 
\hline
Desimal & 0 & 0 & 0 & 0 & 8 & 4 & 2 & 0 & 14 \\ 
\hline
Pangkat & $2^7$ & $2^6$ & $2^5$ & $2^4$ & $2^3$ & $2^2$ & $2^1$ & $2^0$ & $X^{1-7}$ \\ 
\hline
\end{tabular}
\caption{Tabel contoh biner ke desimal.}
\label{table:contoh1}
\end{table}

Mari kita telusuri perlahan-lahan!

\begin{enumerate}
\item Pertama sekali, kita jumlahkan angka pada desimal sehingga menjadi 14. anda lihat angka - angka 


yang menghasilkan angka 14 adalah 8, 4, dan 2!
\item Untuk angka-angka yang membentuk angka 14 (lihat angka yang diarsir), diberi tanda biner “1”, selebihnya diberi tanda “0”.
\item Sehingga kalau dibaca dari kanan, angka desimal 14 akan menjadi 00001110 (terkadang dibaca 1110) pada angka biner nya.
\end{enumerate}


\subsubsection{Mengubah Angka Biner ke Desimal}
Perhatikan contoh! 

\begin{enumerate}
\item 11001101(2) 

\begin{table}[h!]
\centering
\begin{tabular}{ |c|c|c|c|c|c|c|c|c|c| } 
\hline
Biner & 1 & 1 & 0 & 0 & 1 & 1 & 0 & 1 & 11001101 \\ 
\hline
Desimal & 128 & 64 & 0 & 0 & 8 & 4 & 0 & 1 & 205 \\ 
\hline
Pangkat & $2^7$ & $2^6$ & $2^5$ & $2^4$ & $2^3$ & $2^2$ & $2^1$ & $2^0$ & $X^{1-7}$ \\ 
\hline
\end{tabular}
\caption{Tabel contoh biner ke desimal.}
\label{table:contoh2}
\end{table}

Note:

\begin{itemize}
\item Angka desimal 205 didapat dari penjumlahan angka yang di arsir (128+64+8+4+1)
\item Setiap biner yang bertanda “1” akan dihitung, sementara biner yang bertanda “0” tidak dihitung, alias “0” juga.
\end{itemize}

\item 00111100(2) 

\begin{table}[h!]
\centering
\begin{tabular}{ |c|c|c|c|c|c|c|c|c|c| } 
\hline
Biner & 0 & 0 & 1 & 1 & 1 & 1 & 0 & 0 & 00111100 \\ 
\hline
Desimal & 0 & 0 & 32 & 16 & 8 & 4 & 0 & 0 & 60 \\ 
\hline
Pangkat & $2^7$ & $2^6$ & $2^5$ & $2^4$ & $2^3$ & $2^2$ & $2^1$ & $2^0$ & $X^{1-7}$ \\ 
\hline
\end{tabular}
\caption{Tabel contoh biner ke desimal.}
\label{table:contoh2}
\end{table}

Note:
\begin{itemize}
\item Angka desimal 60 didapat dari penjumlahan angka yang di arsir (32+16+8+4)
\item Setiap biner yang bertanda “1” akan dihitung, sementara biner yang bertanda “0” tidak dihitung, alias “0” juga.
\end{itemize}

\item Mengubah Angka Desimal ke Biner 
Untuk mengubah angka desimal menjadi angka biner digunakan metode pembagian dengan angka 2 sambil memperhatikan sisanya. 
Perhatikan contohnya! 
\begin{enumerate}
\item 205(10)
 205 : 2 = 102 sisa 1 
102 : 2 = 51 sisa 0 
51 : 2 = 25 sisa 1
25 : 2 = 12 sisa 1 
12 : 2 = 6 sisa 0 
6 : 2 = 3 sisa 0 
3 : 2 = 1 sisa 1 
1 sebagai sisa akhir “1”
Note :
Untuk menuliskan notasi binernya, pembacaan dilakukan dari bawah yang berarti 11001101(2) 

\item 60(10)
60 : 2 = 30 sisa 0 
30 : 2 = 15 sisa 0 
15 : 2 = 7 sisa 1 











%Lanjutkan diatas tulisan ini
\end{enumerate}
\end{document}