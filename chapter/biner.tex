% Nama Kelompok : Linux
% Kelas : D4 TI 1A
% 1. Kadek Diva Krishna Murti (1174006)
% 2. Duvan Silalahi (1174011)
% 3. Oniwaldus (1174005)
% 4. Choirul Anam (1174004)
% 5. Sri Rahayu (1174015)
% 6. Ilham Habibi (1174028)


\section{Pengertian}

Sejak pertama kali komputer elektronik digunakan, komputer tersebut telah beroperasi dengan menggunakan sistem bilangan biner, yaitu bilangan berbasis dua pada sistem bilangan. Semua kode program dan data pada komputer disimpan serta dimanipulasi dalam format biner yang merupakan kode - kode mesin komputer. Sehingga semua perhitungan – perhitungan yang diolah oleh computer tersebut 

menggunakan aritmatika biner yang hasilnya berupa bilangan hanya memiliki dua kemungkinan nilai, yaitu 0 dan 1. Bilangan Biner atau bilangan berbasis dua atau binary dalam Bahasa Inggris merupakan sebuah penulisan bilangan di mana bilangan – bilangan tersebut hanya menggunakan dua angka, yaitu 0 dan 1. Tidak seperti bilangan desimal yang merupakan

sistem bilangan berbasis 10, sistem bilangan biner berbasis 2. bilangan biner digunakan untuk informasi biner dan juga satuan ukuran besarnya data. Sistem bilangan biner modern detemukan oleh Gottfried Wilhelm Leibniz pada abad ke-17. Sistem bilangan ini merupakan dasar dari semua sistem bilangan berbasis digital. Dari sistem biner , kita dapat mengkonversinya.