
\section(Abstract)
Modem subscriber analog berkecepatan tinggi beroperasi pada kecepatan setinggi 64 kbps baik pada arah downlink maupun uplink menggunakan garis POTS standar ditambah dengan codec yang disempurnakan. Hal ini memungkinkan peningkatan kecepatan upload dan mendukung koneksi pelanggan analog peer-to-peer 56 kbps. Sebuah codec jaringan yang disempurnakan sesuai dengan penemuan ini mendukung jalur POTS baik komunikasi modem berkecepatan tinggi maupun komunikasi ucapan PCM standar. 

\section(definisi)
Modem 56K diperkenalkan di bawah dua standar bersaing yang tidak sesuai. pentingnya persaingan antara penyedia layanan internet dalam proses adopsi.
Bahwa ISP, cenderung mengadopsi teknologi yang lebih banyak pesaing . Hasil ini sangat mencolok mengingat peserta industri mengharapkan koordinasi dalam satu standar atau yang lain.
Berspekulasi tentang peran diferensiasi ISP dalam mencegah pasar mencapai standardisasi sampai organisasi pengaturan standar ikut campur.
Materi pokok dari aplikasi ini terkait erat dengan aplikasi copending berikut yang berhubungan dengan aspek-aspek tertentu dari penemuan ini seperti yang diungkapkan disini dan digabungkan disini sebagai referensi: "Modem kecepatan tinggi dengan pencoba echo-downlink jauh," nomor seri tidak diketahui, oleh Eric M. Dowling dan mengajukan permohonan pada hari yang sama dengan aplikasi ini, 14 Januari 1999.
\subsection(Introduction) Modem V.90 adalah teknologi terbaru yang menawarkan kecepatan koneksi Internet lebih cepat tanpa mengharuskan konsumen berlangganan layanan garis digital yang lebih mahal. Sebelum teknologi V.90, modem secara teoritis dibatasi sekitar 35 Kbps oleh noise kuantisasi yang mempengaruhi konversi analog ke digital (batas praktisnya sebenarnya 33,6 Kbps). Namun, di dunia sekarang ini, dengan meningkatnya fasilitas transmisi digital, aman untuk mengasumsikan bahwa semakin banyak penyedia layanan Internet (ISP) terhubung secara digital baik ke Internet maupun ke kantor pusat perusahaan telepon genggam (KC). Jika demikian, ada koneksi digital yang jelas ke hilir dari modem ISP ke kartu jalur CO yang melayani pengguna dan berisi konverter digital ke analog. Hasil dari koneksi digital ini adalah bahwa konversi analog ke digital (dan oleh karena itu kebisingan kuantisasi) dapat dihindari antara ISP dan CO. Tanpa batasan yang diberlakukan oleh kebisingan kuantisasi, secara teoritis dimungkinkan untuk mencapai kecepatan koneksi hilir hingga 64 Kbps. Praktis, bagaimanapun, ini belum mungkin dilakukan. Hambatan kinerja seperti kuantisasi μ-law mengurangi laju data efektif modem V.90 hingga maksimum 56 Kbps downstream.
\section(Kesimpulan) Dalam penjelasan diatas, modem 56k sangatlah diperlukan dalam mengakses internet. Kita harus berterima kasih kepada pencipta modem 56k. Karena kalau tidak ada dia maka kita tidak akan bisa melakukan chatting di berbagai sosmed dengan cepat. Dialah Dennis Heyes. 
