%kelompok 5 Arsitektur Komputer (Kabel Coaxial)
%Kelas D4 TI 1B
%Tiara Rizki Wulansari (1154026)
%M. A. Faris 1174041
%Evietania Charis Sujadi 1174051
%Iqbal Panggabean 1174063
%Hagan Rowlenstino 1174040
%Irvan Rizkiansyah 1174043


	\subsection{Pengertian dan Fungsi Kabel Coaxial}
	Kabel Coaxial dapat di artikan sebagai suatu media untuk transmisi data dan menyalurkannya melalui sinyal listrik. Kabel Coaxial merupakan sebagai media yang bisa menghubungkan antara satu perangkat dengan perangkat lainnya, karena kabel Coaxial mempunyai kecepatan yang lumayan baik sebagai transmisi data. Fungsi lain dari kabel Coaxial, ialah dapat membagi sinyal broadband atau sebuah sinyal dengan frekuensi tinggi. Berikut adalah beberapa komponen dan bagian pada kabel Coaxial, antara lain :
		\begin{enumerate}
			\item Pada bagian paling dalam kabel COaxial terdapat kabel tembaga yang dimana kabel tersebut berfungsi sebagai media pengantar aliran listrik.
			\item Lapisan plastik, lapisan ini fungsinya menjadi pemisah antara kabel tembaga dan lapisan metal yang membalutnya.
			\item Lapisan metal, lapisan ini sebagai pelindung bagian inti kabel, dan berfungsi pula sebagai pelindung dari pengaruh gelombang elektromagnetik dar luar kabel.
			\item Lapisan plastik terluar, bagian yang melindungi keseluruhan bagian kabel yang berada di dalam kabel.
		\end{enumerate}
	Berikut ini beberapa kelebihan dan kekurangan pada kabel Coaxial :
		\begin{itemize}
			\item Kelebihan
				\begin{enumerate}
					\item Kabel Coaxial relatif memiliki harga yang murah
					\item Kecepatan transmisi yg di miliki kabel Coaxial relatif tinggi, walupun memiliki batasan - batasan jangkauan tertentu
					\item Teknologi yang di terapkan pada jaringan kabel Coaxial masih terbilang sangat umum dan mudah untuk dipahami, dan yang lainnya.
				\end{enumerate}
			\item Kekurangan
				\begin{enumerate}
					\item Dalam urusan pemeliharaan dan perawatan biaya yang dikeluarkan relatif mahal.
					\item Mempunyai sifat yang rentan pada suhu dan temperatur.
					\item Jangkauan sinyal yang sangat terbatas, sehingga memerlukan sebuah repeater lagi untuk menambahkan sinyal jarak jauh, dan yang lainnya.
					\item untuk proses penginstallannya pun kabel coaxial ini termasuk rumit, dikarenakan butuh ketelitian dan kejelian untuk ukuran dari kabel coaxial tersebut
					\item jika kabel ini dipasang di bawah tanahpun rentan sekali terkena gangguan-gangguan fisik yang membuat terputusnya kabel ini, contohnya jika ada gempabumi atau ada tikus tanah dan sebagainya.
				\end{enumerate}
		\end{itemize}
	\subsection {Karakteristik Kabel Coaxial}
	Kabel coaxial memiliki perlindungan intrefensi, dengan maksimal bandwithnya yaitu 10 mbps. Kabel coaxial mempunyai panjang maksimal 500 meter dengan soket atau konektor menggunakan jenis BNC (Bayonet Noval Conector). Harga kabel coaxial relatif lebih murah dibanding kabel fiber optik. Jenis topologi yang biasa diterapkan untuk kabel coaxial ada dua yaitu topologi BUS dan Topologi Ring. Dan untuk instalasi pemasangan kabel coaxial bisa dibilang cukup mudah dan terbilang sederhana.
	
	\subsection {Penerapan Kabel Coaxial Pada Jaringan Komputer}
	Dalam penerapannya, Instalasi pemasangan kabel coaxial harus dilakukan dengan sangat rapi dan hati-hati. Perhitungan kabel jaringan coaxial harus diukur dengan sempurna karena jika salah dalam perhitungan ukuran dapat mengakibatkan rusaknya NIC (Network Interface Card) yang dipergunakan. Selain dapat merusak NIC, Kesalahan pengukuran kabel jaringan coaxial dalam instalasi pemasangan juga memberikan dampak pada kinerja jaringan itu sendiri yang akan terhambat karena jaringan tidak mencapai kemampuan maksimalnya.Ada beberapa hal yang perlu diperhatikan dalam instalasi pemasangan kabel coaxial untuk mendapatkan hasil yang sempurna:
		\begin{itemize}
			\item Kontinuitas konduktor utama kabel coaxial harus dalam kondisi baik dan terpelihara
			\item Pada sambungan kabel coaxial harus ketat sehingga kabel tersebut tetap bersifat homogen seperti pada kondisi awal
			\item Redaman yang didapatkan harus bisa tetap pada angka nol atau sekecil-kecilnya
			\item Hasil dari pekerjaan sambungan kabel coaxial tersebut harus benar-benar rapi.
	

	
Artikel yang dirangkum dari sebuah buku \cite{syafrizal2005pengantar}
Dari sebuah artikel yang dirangkum \cite{kelik2003pengantar}
Dari sebuah artikel yang dirangkum \cite{beveridge1995method}
\end{itemize}
