%kelompok 5 Arsitektur Komputer (Kabel Coaxial)
%Kelas D4 TI 1B
%Tiara Rizki Wulansari (1154026)
%M. A. Faris 1174041
%Evietania Charis Sujadi 1174051
%Iqbal Panggabean 1174063
%Hagan Rowlenstino 1174040
%Irvan Rizkiansyah 1174043


\section(Kabel Coaxial)
Di dalam dunia IT khususnya Networking, untuk membentuk suatu jaringan, baik itu bersifat LAN (Local Area Network), maupun WAN (Wide Area Network), kita memerlukan media baik hardware maupun software. Beberapa media hardware yang penting di dalam membangun suatu jaringan adalah kabel atau perangkat Wi-Fi, ethernet card, hub atau switch, repeater, bridge, atau router dll. ada beberapa jenis kabel yang banyak digunakan dan menjadi standart untuk membangun atau sebagai penggunaan komunikasi data dalam jaringan komputer. Namun perlu diingat bahwa hampir 85 persen kegagalan yang terjadi pada jaringan komputer disebabkan adanya kesalahan pada media komunikasi yang digunakan termasuk kabel. kabel coaxial salah satunya adalah jenis kabel yang sering digunakan untuk membangun jaringan dalam lingkup LAN (Local Area Network).
Dikenal dua jenis tipe kabel coaxial yang digunakan untuk jaringan komputer, yaitu:
	\begin{itemize}
		\item Thick coaxial (mempunyai diameter lumayan besar) Thick coaxial cable sudah dispesifikasikan dengan berdasarkan standar IEEE 802.3 10 BASE 5, yang rata-rata diameternya adalah sekitar 12cm, yang biasanya diberikan warna kuning. Kabel ini juga biasa disebut dengan standard ethernet atau juga bisa dipanggil dengan thick Ethernet, atau yang biasa dikenal dengan ThickNet dan yellow cable. Kabel jenis ini mempunyai spesifikasi dan aturan sebagai berikut :
			\begin{itemsize}
				\item Setiap ujungnya harus diterminasi menggunakan terminator rakitan sebesar 50-ohm.
				\item Peralatan yang terhubung maksimal 3 segment.
				\item Ada pemancar tambahan di setiap pemancar jaringannya.
				\item Setiap segment tadi maksimal berisi 100 perangkat jaringan, sudah termasuk juga repreater.
				\item Untuk kabelnya, maksimum sekitar 500 meter per segment nya.
				\item Jarak antar segment tidak boleh lebih dari 1500 meter.
				\item Ground harus sudah terpasang di setiap segment.
				\item Jarak terjauh untuk pencabang dari kabel utama ke device hanya sekitaran 5 meter saja.
				\item Setiap pencabang paling banyak hanya boleh berjarak sekitar 2,5 meter.
			\end{itemsize}
			
		\item Thin coaxial (mempunyai diameter lebih kecil). Thin Coaxial ini biasa digunakan untuk transciver di banyak radio amatir yang hanya memerlukan output atau pengeluaran daya yang kecil. Agar dapat digunakan sebagai jaringan, kabel ini harus memenuhi standar IEEE 802.3 10BASE2,yang diameter rata-ratanya kurang lebih 5mm dengan warna hitam atau warna gelap yang lain dan setiap perangkat di sambungkan ke BNCT-connector. Jika ingin kabel ini diimplementaasikan dengan T-Connector dan terminator di dalam sebuah jaringan, maka harus mengikuti aturan-aturan ini: 
			\begin{itemsize}
				\item Seperti biasa, tiap ujungnya diberikan terminator sebesar 50-ohm.
				\item Panjang kabel per segment nya kira-kira 185 meter.
				\item Maksimal dapat terkoneksi 30 device per segment.
				\item Kartu jaringannya dapat menggunakan transceiver yang sudah terpasang, kecuali untuk reapreater.
				\item Maksimal 3 segment yang berhubungan satu dengan yang lainnya.
				\item Sebaiknya menggunakan satu ground di setiap segment nya.
				\item Panjang minimal T-connector minimal 0,5 meter.
				\item Panjang maksimum kabel per segment adalah 555 meter.
				\item dapat menampung maksimum 30 device per segmentnya.
			\end{itemsize}
			
	\begin{itemize}
		\item Thick coaxial (mempunyai diameter lumayan besar) Thick coaxial cable sudah dispesifikasikan dengan berdasarkan standar IEEE 802.3 10 BASE 5, yang rata-rata diameternya adalah sekitar 12cm, yang biasanya diberikan warna kuning. Kabel ini juga biasa disebut dengan standard ethernet atau juga bisa dipanggil dengan thick Ethernet, atau yang biasa dikenal dengan ThickNet dan yellow cable. Kabel jenis ini mempunyai spesifikasi dan aturan sebagai berikut :

	\subsection{Pengertian dan Fungsi Kabel Coaxial}
	Kabel Coaxial dapat di artikan sebagai suatu media untuk transmisi data dan menyalurkannya melalui sinyal listrik. Kabel Coaxial merupakan sebagai media yang bisa menghubungkan antara satu perangkat dengan perangkat lainnya, karena kabel Coaxial mempunyai kecepatan yang lumayan baik sebagai transmisi data. Fungsi lain dari kabel Coaxial, ialah dapat membagi sinyal broadband atau sebuah sinyal dengan frekuensi tinggi. Berikut adalah beberapa komponen dan bagian pada kabel Coaxial, antara lain :
		\begin{enumerate}
			\item Pada bagian paling dalam kabel COaxial terdapat kabel tembaga yang dimana kabel tersebut berfungsi sebagai media pengantar aliran listrik.
			\item Lapisan plastik, lapisan ini fungsinya menjadi pemisah antara kabel tembaga dan lapisan metal yang membalutnya.
			\item Lapisan metal, lapisan ini sebagai pelindung bagian inti kabel, dan berfungsi pula sebagai pelindung dari pengaruh gelombang elektromagnetik dar luar kabel.
			\item Lapisan plastik terluar, bagian yang melindungi keseluruhan bagian kabel yang berada di dalam kabel.
		\end{enumerate}
	Berikut ini beberapa kelebihan dan kekurangan pada kabel Coaxial :
		\begin{itemize}
			\item Kelebihan
				\begin{enumerate}
					\item Kabel Coaxial relatif memiliki harga yang murah
					\item Kecepatan transmisi yg di miliki kabel Coaxial relatif tinggi, walupun memiliki batasan - batasan jangkauan tertentu
					\item Teknologi yang di terapkan pada jaringan kabel Coaxial masih terbilang sangat umum dan mudah untuk dipahami, dan yang lainnya.
				\end{enumerate}
			\item Kekurangan
				\begin{enumerate}
					\item Dalam urusan pemeliharaan dan perawatan biaya yang dikeluarkan relatif mahal.
					\item Mempunyai sifat yang rentan pada suhu dan temperatur.
					\item Jangkauan sinyal yang sangat terbatas, sehingga memerlukan sebuah repeater lagi untuk menambahkan sinyal jarak jauh, dan yang lainnya.
					\item untuk proses penginstallannya pun kabel coaxial ini termasuk rumit, dikarenakan butuh ketelitian dan kejelian untuk ukuran dari kabel coaxial tersebut
					\item jika kabel ini dipasang di bawah tanahpun rentan sekali terkena gangguan-gangguan fisik yang membuat terputusnya kabel ini, contohnya jika ada gempabumi atau ada tikus tanah dan sebagainya.
				\end{enumerate}
		\end{itemize}
	\subsection {Karakteristik Kabel Coaxial}
	Kabel coaxial memiliki perlindungan intrefensi, dengan maksimal bandwithnya yaitu 10 mbps. Kabel coaxial mempunyai panjang maksimal 500 meter dengan soket atau konektor menggunakan jenis BNC (Bayonet Noval Conector). Harga kabel coaxial relatif lebih murah dibanding kabel fiber optik. Jenis topologi yang biasa diterapkan untuk kabel coaxial ada dua yaitu topologi BUS dan Topologi Ring. Dan untuk instalasi pemasangan kabel coaxial bisa dibilang cukup mudah dan terbilang sederhana.
	
	\subsection {Penerapan Kabel Coaxial Pada Jaringan Komputer}
	Dalam penerapannya, Instalasi pemasangan kabel coaxial harus dilakukan dengan sangat rapi dan hati-hati. Perhitungan kabel jaringan coaxial harus diukur dengan sempurna karena jika salah dalam perhitungan ukuran dapat mengakibatkan rusaknya NIC (Network Interface Card) yang dipergunakan. Selain dapat merusak NIC, Kesalahan pengukuran kabel jaringan coaxial dalam instalasi pemasangan juga memberikan dampak pada kinerja jaringan itu sendiri yang akan terhambat karena jaringan tidak mencapai kemampuan maksimalnya.Ada beberapa hal yang perlu diperhatikan dalam instalasi pemasangan kabel coaxial untuk mendapatkan hasil yang sempurna:
		\begin{itemize}
			\item Kontinuitas konduktor utama kabel coaxial harus dalam kondisi baik dan terpelihara
			\item Pada sambungan kabel coaxial harus ketat sehingga kabel tersebut tetap bersifat homogen seperti pada kondisi awal
			\item Redaman yang didapatkan harus bisa tetap pada angka nol atau sekecil-kecilnya
			\item Hasil dari pekerjaan sambungan kabel coaxial tersebut harus benar-benar rapi.
	

	
Artikel yang dirangkum dari sebuah buku \cite{syafrizal2005pengantar}
Dari sebuah artikel yang dirangkum \cite{kelik2003pengantar}
Dari sebuah artikel yang dirangkum \cite{beveridge1995method}
\end{itemize}
