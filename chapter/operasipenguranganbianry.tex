PENGURANGAN BILANGAN BINER

keadaan yang muncul dipengurangan bil biner
(0-0, 1-0, 1-1)

Dimana pada keadaan

	0-0 = 0
	0-1 = 1 meminjam 1 (apabila masih ada angka disebelah kirinya)
	1-0 = 1
	1-1 = 0
	
maksud dari peminjaman angka adalah meminjam ketika angka yang disebelah kiri lebih besar agar dapat mencukupi ketika melakukan
pengurangan.

keadaan yang sama pun akan berlaku pada bilangan desimal.
contohnya pada bilangan desimal:

	45-40 = 5 (tidak perlu meminjam karena nilainya mencukupi)
				pengurangan yang satu ini tidak perlu melakukan peminjaman angka karena angkanya sudah mencukupi dalam melakukan 
				pengurangan
				
	20-19 = 1 (angka 0 harus meminjam 1 dari angka 2 supaya dapat menguranginya)
				caranya: ketika angka 0 meminjam 1 angka dari angka 2 maka nilainya menjadi 10-9.
				
Contoh Soal

1100010-110111

Lihatlah operasi pengurangan bilangan binary berikut,
 
Bagaimana cara mengerjakannya?
Sama seperti cara operasi pengurangan baisanya tetapi ada kondisi teretentu yang harus diperhatikan seperti penjelasan sebelumnya.
Mari kita kerjakan!
0-1 = 1 pengoperasian pengurangan mengharuskan meminjam
0-1 = 1  pengoperasian pengurangan mengharuskan meminjam
1-1 = 0
1-0 = 1 pengoperasian pengurangan mengharuskan meminjam
1-1 = 0
0-1 = 1 pengoperasian pengurangan mengharuskan meminjam
0 dikarenakan habis dipinjam dari 1 menjadi 1
