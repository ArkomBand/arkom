/section{Perkalian}
/subsection(Pengertian dasar perkalian biner)
Perkalian dalam biner mirip dengan pasangan desimalnya. Dua angka A dan B dapat dikalikan dengan produk parsial: 
untuk setiap digit di B, produk dari digit di A dihitung dan ditulis pada baris baru, bergeser ke kiri sehingga 
garis digit paling kanannya naik dengan angka di B yang bekas. Jumlah semua produk parsial ini memberikan hasil akhir.

/subsection{definisi hexadesimal}
Sistem angka heksadesimal, yang juga dikenal sebagai hex, adalah sistem angka yang terdiri dari 16 simbol (dasar 16). 
Sistem angka standar disebut desimal (basis 10) dan menggunakan sepuluh simbol: 0,1,2,3,4,5,6,7,8,9. Heksadesimal 
menggunakan angka desimal dan mencakup enam simbol tambahan. Tidak ada simbol yang berarti sepuluh, atau sebelas, 
jadi simbol ini diambil dari alfabet Inggris: A, B, C, D, E dan F. Heksadesimal A = desimal 10, dan heksadesimal F = desimal 15

/subsection{sistem bilangan hexadesimal terhadap desimal}
Heksadesimal atau sistem bilangan basis 16 adalah sebuah sistem bilangan yang menggunakan 16 simbol. Berbeda dengan sistem bilangan desimal, simbol yang digunakan dari sistem ini adalah angka 0 sampai 9, ditambah dengan 6 simbol lainnya dengan menggunakan huruf A hingga F. Sistem bilangan ini digunakan untuk menampilkan nilai alamat memori dalam pemrograman komputer.

/subsection{Pengertian luas hexadesimal}
Dalam matematika dan komputasi, heksadesimal (juga basis 16, atau heks) adalah sistem angka posisional dengan radix, atau basis, dari 16. Ini menggunakan enam belas simbol yang berbeda, paling sering simbol 0-9 untuk mewakili nilai nol sampai sembilan, dan A, B, C, D, E, F (atau alternatif a, b, c, d, e, f) untuk mewakili nilai sepuluh sampai lima belas.
Angka heksadesimal banyak digunakan oleh perancang dan pemrogram sistem komputer. Karena setiap digit heksadesimal mewakili empat digit biner (bit), ini memungkinkan representasi biner yang lebih ramah manusia. Satu digit heksadesimal mewakili nibble (4 bit), yang merupakan setengah dari oktet atau byte (8 bit). Sebagai contoh, satu byte dapat memiliki nilai mulai dari 00000000 sampai 11111111 dalam bentuk biner, tapi ini mungkin lebih mudah direpresentasikan sebagai 00 sampai FF dalam heksadesimal.
Dalam konteks non-pemrograman, subskrip biasanya digunakan untuk memberi radix, misalnya nilai desimal 10.995 akan dinyatakan dalam heksadesimal sebagai 2AF316. Beberapa notasi digunakan untuk mendukung representasi heksadesimal dari konstanta dalam bahasa pemrograman, biasanya melibatkan awalan atau akhiran. Awalan "0x" digunakan dalam bahasa C dan bahasa terkait, di mana nilai ini dapat dinotasikan sebagai 0x2AF3.
/subsection {contoh perkalian biner}
/subsubsection {Perkalian dengan 4}
Dari Tabel 1 dapat ditentukan
Satuan Hasil Perkalian dengan 4 (SHP4)
k 0 2 4 6 8
SHP4 0 8 6 4 2
Jika k genap maka :
SHP4 = 10- k
k 1 3 5 7 9
SHP4 4 2 0 8 6
Jika k ganjil maka:
SHP4 = 15- k
Sehingga cara mudah menentukan hasil
perkalian bilangan n digit dengan 4 :
1. Untuk angka terkanan =
Nilai satuan pada : 10- k, k genap
Nilai satuan pada : 15- k, k ganjil
Jika memuat puluhan simpan sebagai
”simpanan”
2. Untuk angka di sebelah kirinya =
Nilai satuan pada : (9- k)+”s” dari
tetangganya, k genap
Nilai satuan pada : (9- k) + “s” dari
tetangganya + 5, k ganjil
Jika dari langkah 1 diperoleh
“simpanan” maka “simpana’ yang ada
ditambahkan pula.
Jika hasilnya memuat puluhan simpan
sebagai ”simpanan”
3. Ulangi langkah 2 sampai digit ke n
4. Untuk digit ke (n+1) =
”s” dari digit ke n + ”simpanan”-1
Contoh :
( 1 ) 4765 X 4 = ?
Penyelesaian :
C = ”simpanan”, H = hasil
k OPERASI C H
5 15-5 1 0
6 (9-6)+2+1 0 6
7 (9-7)+3+5+0 1 0
4 (9-4)+3+1 0 9
0 2-1+0 0 1
Jadi 4765 X 4 = 19060
( 2 ) 87645912 X 4 = ?
Penyelesaian :
C = ”simpanan”, H = hasil
k OPERASI C H
2 10-2 0 8
1 (9-1)+1+5+0 1 4
9 (9-9)+0+5+1 0 6
5 (9-5)+4+5+0 1 3
4 (9-4)+2+1 0 8
6 (9-6)+2+0 0 5
7 (9-7)+3+5+0 1 0
8 (9-8)+3+1 0 6
0 4-1+0 0 3
Jadi 87645912 X 4 = 360583648
Putut Sriwasito (Perkalian Biner Bilangan N Digit Dengan 3, 4, 5 dan 6)
41

