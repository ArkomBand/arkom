Sejarah Perkembangan Wi-Fi (Wireless Fidelity)
Wireless Fidelity adalah satu standart wireless networking tanpa kabel. teknologi spesifikasi ini emiliki standart yang ditetapkan oleuh sebuah institusi internasional yang bernama IEEE  (Insitute of Electrical and Electronic Engineers). Di tahun 1997 sebuah lembagaindependen bernama IEEE membuat standart WLAN pertama yang diberi kode 802.11. dapat bekerja pada frekuensi 2,4GHz dengan kecepatan transfer data 2Mbps.
Empatsejarah singkat perkembangan protokol Wireless fidelity:
1. pada bulan juli 1999, IEEE merilis spesifikasi baru yang bernama 802.11b. dengan kecepatan transfer data maksimal11Mbps.
2. pada waktu yang hampir bersamaan institute of electrical and electronic engineers kembali membuat spesifikasi 802.11a yang menggunakan teknik berbeda. Frekuensi yang  digunakan 5GHz, dan mendukung kecepatan transfer data dengan kecepatan sampai 54Mbps
3.pada tahun 2002. institute of electrical and electronic engineers membuat spesifikasi baru yang dapat menggabungkan kelebihan antara 802.11b dengan 802.11a. spesifikasi baru  yang diberi kode 802.11g ini bekerja pada frekuensi 2,4GHz dengan kecpatan transfer data maksimal 54Mbps.
4.di tahun 2006 institute of electrical and electronic engineers mengembangkan teknologi terbarunya dengan menggabungkan teknologi 802.11b dengan 802.11g menjadi 802.11n. teknologi ini dikenal dengan istilah MIMO (Multiple Input Multiple Output) teknologi wireless fidelity terbaru
 
SEJARAH WI-FI
Sejarah Wi-Fi berasal dari sebuah istilah puluhan tahun Hi-Fi yang terdiri dari jenis output yang dihasilkan olehkualitas sound system. Teknollgi Wireless Fidelity berspesifikasi standart Institute of Electrical and Electronic Engineers atau yang disingkat dengan IEEE 802.. termasuk 802.11a, 802.11b, dan 802.11g. Wireless Fidelity atau yang disingkat dengan Wi-Fi adalah hanya istilah produk teknologi yang dipromosikan oleh WIFI Alliance.
Sejarah Wireless Fidelity itu sendiri dimulai ketika tahun 1985 dari hasil kerja keras insinyur Amerika dengan pengguna Teknologi penyebaran spektrum radio yang digunakan dalam Wi-Fi. Wireless LAN atau Wi-Fi dibuat dan tersedia untuk umum di Amerika Serikat di tahun 1985, tidak ada lisensi dari komisi komunikasi federal (FCC). Kemudian Michael Marcus mengusulkan untuk menggunakan wireless LAN dan teknologi radio untuk publik.
