\subsection(SEJARAH)
Penemuan ini,  telah dirancang untuk memecahkan masalah dan objeknya adalah untuk menyediakan pembagi yang dapat melakukan pembagian dengan pembagi 
dan semua pembagi menjadi bilangan heksadesimal. Pembagi dari penemuan ini dibuat untuk menyelaraskan digit dari pembagi normalisasi normalisasi di muka 
dengan secara selektif menggunakan fungsi pergeseran dan fungsi pergeseran yang tepat yang dibangun pada pemilih, dan kemudian menentukan hasil pembagian
heksadesimal dengan mengulangi proses dengan menentukan nomor kali.
Penemuan pertama pembagi yang terkait dengan penemuan ini dilengkapi dengan rangkaian normalisasi pertama untuk memasukkan data dari data floating point pembagi yang basisnya 16 dan menormalisasinya berdasarkan basis di atas, 
rangkaian normalisasi kedua untuk memasukkan data dari Pembagi adalah data floating point yang basisnya adalah 16 dan menormalisasinya berdasarkan basis di atas, rangkaian pembagi, 
dan pemilih untuk memasukkan data mantissa dari pembagi dari rangkaian normalisasi pertama, 
sisa data dari rangkaian pemisah dan sinyal siklus divisi yang menunjukkan siklus divisi, dan ketika sinyal siklus divisi menunjukkan siklus pertama,

\subsection{Bilangan Biner}
Sejak pertama kali komputer elektronik digunakan, komputer beroperasi dengan menggunakan bilangan biner, yaitu bilangan dengan basis 2 pada sistem bilangan. Semua kode program dan data pada komputer disimpan serta dimanipulasi dalam format biner yang merupakan kode-kode mesin komputer. Sehingga semua per-hitungannya diolah menggunakan aritmatik biner, yaitu bilangan yang hanya memiliki nilai dua kemungkinan yaitu 0 dan 1 dan sering disebut sebagai bit (binary digit atau dalam arsitektur elektronik biasa disebut sebagai digital logic. Representasi bilangan biner bas dilihat disamping ini. Posisi sebuah angka akan menentukan berapa bobot nilainya. Posisi paling depan (kiri) sebuah bilangan memiliki nilai yang paling besar sehingga disebut sebarai MSB (Most Significant Bit), dan posisi paling belakang (kanan) sebuah bilangan memiliki nilai yang paling kecil sehinggal disebut sebagai LSB (Leased Significant Bit).

Contoh: reprentasi bilangan dengan basis biner:
101102 = 1*2^4 + 0*2^3+1*2^1+0*2^0=2210

\subsection{Bilangan Heksadesimal}
Bilangan heksadesimal atau biasa disebut heksa saja, berbasis 16 memiliki nilai yang disimbolkan dengan 0, 1, 2, 3, 4, 5, 6, 7, 8, 9, a, b, c, d, e, f. Adanya bilanagn ini dikarenakan operasi bilangan biner untuk data yang lebih besar akan menjadi susah, hingga bilangan ini sering digunakan untuk menggambarkan memori computer atau intruksi. Setiap digit bilangan heksa mewakili 4 bit bilangan biner, dan 2 digit bilangan heksadesimal mewakili satu byte.
