\section {Hexadecimal dan Binary}
\subsection {Hexadecimal}
Hexadecimal atau juga bisa disebut dengan sistem bilangan basis 16 adalah sistem bilangan yang menggunakan 16 simbol. berbeda dengan sistem bilangan desimal. Simbol yang digunakan dari sistem ini adalah angka 0 sampai 9, ditambah dengan 6 simbol lainnya dari huruf A sampai F. Sistem bilangan ini digunakan untuk menampilkan nilai alamat memori dalam pemrograman komputer.
ciri ciri bilangan hexadecimal adalah :
\begin{itemize}
	\item Bilangan hexadecimal memiliki bilangan basis 16
	\item Mempunyai 16 kemungkinan digit yang terbentuk
	\item Menggunakan angka dan huruf seperti (0-9) dan (A-F),huruf A-F untuk merepresentasikan 10-15
	\item Setiap digitnya merupakan bilangan pangkat 16 yang diasosiasikan berdasarkan posisinya.
\end{itemize}

\subsection {Penjumlahan Hexadecimal}
penjumlahan bilangan hexadecimal harus dijumlahkan berurutan dari digit yang paling kanan. Bagi 2 bilangan yang dijumlahkan, kalau hasil dari penjumlahannya lebih dari 15 maka akan menjadi carry 1, lalu hasil dari penjumlahan tersebut dikurangi 16 yang akan menjadi hasil dari penjumlahan hexadecimal. Perhatikan contoh dibawah

153(16) + 234(16) = .......... (16) 
Langkah-langah penyelesainnya
153 
234 
---- (+)

1. 3 + 4 = 7
2. 5 + 3 = 8
3. 1 + 2 = 3

Karena tidak terdapat carry, maka 153(16) + 234(16) = 387(16)


\subsection {Biner}
Komputer menggunakan bit (digit biner, sebuah keadaan elektronik yang mewakili angka nol dan satu) untuk menunjukkan nilai. kami mewakili bilangan biner seperti itu dengan menggunakan angka 0 dan 1 sistem nilai 2 tempat. Sistem bilangan biner ini seperti sistem desimal kecuali bahwa posisi (kanan ke kiri) adalah 1, 2, 3, 4, 8, 16 (dan kekuatan yang lebih tinggi dari 2) bukan 1, 10, 100, 1000, 10000 (kekuatan 10). Sebagai contoh , bilangan biner 1101 dapat diartikan sebagai angka desimal 13.
\begin{verbatim}
		1			1			0			1
	one 8	  +	 one 4	  +	  one 2    +   one 1 	= 13
	\end{verbatim}
Nomor pada  biner begitu panjang sehingga mereka canggung  untuk membaca dan juga menulis. Misalnya saja, di butuhkan 8 bit yaitu  11111010 untuk mewakili angka 250, atau 16 bit yaitu 111010100110000 untuk mewakili bilangan desimal 30000.


\subsection {Penjumlahan Biner}
Bilangan biner dapat juga dilakukan penambahan, pengurangan, perkalian dan pembagian. Kali ini, kelompok ini akan membahas tentang penjumlahan bilangan biner. Berikut adalah tata cara atau aturan penjumlahan bilangan biner :
Untuk penjumlahan operasi biner biasa, adalah operasi yang bersifat komutatif, karena untuk sembarang bilangan seperti x dan y, berlaku y+x = x+y. Operasi penjumlahan bersifat asosiatif, karena untuk x,y,dan z yang sembarang, berlaku x+(y+z)=(x+y)+z. Identitas yang berlaku di operasi penjumlahan ini adalah 0 (nol). Dan invers penjumlahan untuk bilangan sembarang x adalah -x, karena x+(-X)=0.

A0 + B0 = ∑0 + Cout
Di dalam melakukan sebuah penjumlahan biasanya akan selalu melibatkan penjumlahan dengan carry in.

Di dalam Penjumlahan Biner sendiri Ada 4 kondisi yaitu
(0+0, 1+0, 0+1, 1+1) dimana

0 + 0 = 0
1 + 0 = 1
0 + 1 = 1
1 + 1 = 0 (carry out) 

Untuk maksud dari Carry out sendiri adalah hasilnya tidak bisa memuat lebih dari 1 digit. Tetapi  hasil tadi disimpan kedalam kolom sebelah yang lebih tinggi nilainya.

Berikut contoh pada sebuah bilangan desimal
7 + 2 = 9 (CarryOut = 0)
8 + 15 = 23 (CarryOut = 0)
Maksud dari Carry Out itu sendiri adalah penyimpangan angka, berdasarkan contoh di atas. 7 + 2 = 9 CarryOut = 0 sebab tidak ada bilangan yang di simpan. 
