%Fiber Optik (Arsitektur Komputer)
%Kelas : D4 TI 1B
%Khadijah Hasanah Putri Harahap 1174022
%Liyana Majdah Rahma 1174039
%Luthfi Muhammad Nabil 1174035
%Nisrina Aulia Firdaus 1174098
%Salwaa Tania 1174047
%Septia Rahayu 1174044
%Diana Satima Gistivani 1154018

\section{Fiber Optic}
\begin{flushleft}
Fiber Optic merupakan sebuah kabel tembus pandang berbahan kaca atau plastik yang halus dan kecil yang digunakan untuk mentransmisikan sinyal cahaya dari satu tempat ke tempat lain, Sumber cahaya dari Fiber Optic biasanya menggunakan cahaya Laser atau LED. Ukuran diameter dari kabel ini kurang lebih sekitar 125 mikrometer atau sekitar 1/8 mm. Kabel Fiber Optic sendiri biasa dipakai untuk kepentingan Jaringan telepon atau Koneksi Internet.
\end{flushleft}
\begin{flushleft}
Gelombang cahaya pada kabel Fiber Optic dipantulkan dari satu ujung ke ujung yang lain tanpa menggunakan perantara apapun, radius dari pantulan cahaya Fiber Optic bisa mencapai 50 Kilometer sedangkan jika memakai perantara seperti repeater dapat mencapai 100 Kilometer. kabel Fiber Optic memiliki daya pantul cahaya yang sangat tinggi sehingga membuat cahaya pada kabel tidak mudah meredup atau melemah dibagian tengah kabel. 
\end{flushleft}
\subsection 
Teknologi serat optik (fiber optic) ini akan memberikan kemungkinan yang lebih baik bagi jaringan telekomunikasi, terutama dalam hal telekomunikasi data. 
Serat optik (fiber optic) adalah salah satu media transmisi yang dapat menyalurkan informasi dengan kapasitas besar dengan tingkat keandalan (performance) yang tinggi. 
Beda dengan media transmisi lain, pada teknologi serat optik (fiber optic) ini, gelombang pembawanya tak lagi merupakan gelombang elektromagnetik (microwave) atau listrik, 
tapi merupakan sinar atau cahaya laser. 
Kabel serat optik (fiber optic) mampu melayani transfer data dengan kecepatan tinggi dalam waktu yang relatif singkat dan bentuk fisi yang relatif kecil dan ringan.

