\section{pengertian hexadecimal}
	Hexadecimal adalah sebuah sistem bilangan yang menggunakan sebuah simbol.Dalam hexadecimal Terdapat beberapa simbol yang bisa digunakan di sistem bilangan ini.Berbeda dengan bilangan decimal.hexadecimal menggunakan angka 0 sampai 1, di bilangan hexadecimal ini tidak menggunakan angka semua melainkan ada beberapa simbol yang menggunakan huruf.jumlah simbol yang yang berasal dari angka 1 sampai 9 berjumlah 16 simbol, ditambah dengan 6 simbol lainnya yang menggunakan huruf dari A sampai F.Hexadecimal bisa digunakan untuk menampilkan nilai alamat memori dan pemrograman komputer.Teknik penjumlahan dan pengurangan pada bilangan hexadecimal hampir sama dengan penjumlahan dan pengurangan pada bilangan biner,octal dan decimal, tetapi jika terjadi carry 1 atau borrow 1, maka angka 1 tersebut bernilai 16. Carry akan terjadi apabila penjumlahan lebih dari 15 misalnya 8+8=10. Sedangkan borrow terjadi apabila angka yang dikurangi lebih kecil dari pengurang, misalnya 45-6=.
\section{operasi penjumlahan pada bilangan hexadesimal}
penjumlahan bilangan hexadesimal dapat dilakukan secara sama dengan penjumlahan bilangan octal, dengan langkah-langkah sebagai berikut: 1) tambahkan masing-masing kolom secara desimal, 2) rubah dari hasil desimal ke hexadesimal, 3) tuliskan hasil dari digit paling kanan dari hasil hexadesimal, 4) jika hasil penjumlahan setiap kolom terdiri dari dua digit, maka digit paling kiri merupakan carry of untuk penjumlahan pada kolom selanjutnya.
\section{operasi pengurangan pada bilangan hexadesimal}
pengurangan mudah diselesaikan jika dikerjakan dengan rapi yaitu memperhatikan lajur-lajur perseratusan, persepuluhan, satuan, puluhan, ratusan, dan sebagainya. untuk menyelesaikan pengurangan bilangan hexadesimal, ikuti langkah-langkah ini: 1)tulis kedua bilangan bersusun ke bawah, sejajarkan sehingga koma hexadesimal membentuk baris lurus, 2) tambahkan nol agar bilangan memiliki panjang yang sama, 3) kemudian kurangkan, jangan lupa mencantumkan koma hexadesimal pada jawabannya.
\section{operasi perkalian pada bilanagn decimal}
cara mengoperasikan perkalian bilangan hexadecimal sama seperti opersi perkalian pada bilanagan decimal. Caranya sebagai berikut :
	1.Kalikan masing-masing kolom secara decimal.
	2.Kemudian ubah dari hasil decimal ke hexadecimal.
	3.Tuliskan hasil dari digit paling kanan dari hasil hexadecimal.
	4.Jika hasil perkalian tiap-tiap kolom terdiri dari 2 digit,maka digit yang berada pada posisi yang paling kiri merupakan carry of untuk ditambahkan pada pada hasil perkalian kolom berikutnya.
\section{operasi aritmatika pada bilangan hexadesimal}
penjumlahan bilangan hexadesimal dapat dilakukan secara sama dengan penjumlahan bilangan octal, dengan langkah-langkah sebagai berikut: 1) tambahkan masing-masing kolom secara desimal, 2) rubah dari hasil desimal ke hexadesimal, 3) tuliskan hasil dari digit paling kanan dari hasil hexadesimal, 4) jika hasil penjumlahan setiap kolom terdiri dari dua digit, maka digit paling kiri merupakan carry of untuk penjumlahan pada kolom selanjutnya.
\section {operasi pembagian pada hexadecimal}
Pembagian pada bilangan Hexadecimal sama seperti pembagian pada bilangan decimal. adapun langkah pembagian pecahan decimal dengan cara mudah, yaitu :
	Contoh : 0,625 : 0,25
	1.No Reken Koma ( tidak menghiraukan tanda koma), sehingga bilangan 0,625 menjadi 625 dan 0,25 menjadi 25.
	2.Bagi bilangan tanpa koma, 625 : 25 = 25.
	3.Hitung banyaknya angka di belakang koma bilangan yang dibagi dengan banyaknya angka di belakang bilangan pembagi, jika hasil pengurangannya adalah positif 1,2,3 dst, maka hasil pembagian pada point 2 masih harus dibagi 10, 100, 1000 dst atau masih harus ada 1,2,3 angka di belakang koma, dan jika hasil pengurangan adalah 0 maka hasil pembagian pada point 2 tetap. lalu jika hasil pengurangannya adalah negatif (-1,-2,-3 dst, maka hasil pembagian pada point masih harus dikalikan dengan 10, 100, 1000, dst). Terdapat 3 angka di belakang koma pada bilangan yang dibagi dan terdapat 2 angka pada bilangan pembagi.
	4.Kurangkan banyaknya angka di belakang koma bilangan yang dibagi dengan banyaknya angka di belakang koma pembagi, yaitu 3-2 = .
	5.Hasil pengurangan menunjukkan bahwa hasil pembagian pada point 2 harus dibagi 10 atau harus terdapat satu angka di belakang koma pada bilangin 25, yaitu 2,5.