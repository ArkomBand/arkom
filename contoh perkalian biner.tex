/section{Perkalian}
 
 /subsection(Pengertian dasar perkalian biner)
  Perkalian dalam biner mirip dengan pasangan desimalnya. Dua angka A dan B dapat dikalikan dengan produk parsial: 
  untuk setiap digit di B, produk dari digit di A dihitung dan ditulis pada baris baru, bergeser ke kiri sehingga 
  garis digit paling kanannya naik dengan angka di B yang bekas. Jumlah semua produk parsial ini memberikan hasil akhir.
  
 /section{definisi hexadesimal}
 /subsection{definisi hexadesimal}
  Sistem angka heksadesimal, yang juga dikenal sebagai hex, adalah sistem angka yang terdiri dari 16 simbol (dasar 16). 
  Sistem angka standar disebut desimal (basis 10) dan menggunakan sepuluh simbol: 0,1,2,3,4,5,6,7,8,9. Heksadesimal 
  menggunakan angka desimal dan mencakup enam simbol tambahan. Tidak ada simbol yang berarti sepuluh, atau sebelas, 
 -jadi simbol ini diambil dari alfabet Inggris: A, B, C, D, E dan F. Heksadesimal A = desimal 10, dan heksadesimal F = desimal 15
 +jadi simbol ini diambil dari alfabet Inggris: A, B, C, D, E dan F. Heksadesimal A = desimal 10, dan heksadesimal F = desimal 15
 
 /subsection{sistem bilangan desimal}
 +Heksadesimal atau sistem bilangan basis 16 adalah sebuah sistem bilangan yang menggunakan 16 simbol. Berbeda dengan sistem bilangan desimal, simbol yang digunakan dari sistem ini adalah angka 0 sampai 9, ditambah dengan 6 simbol lainnya dengan menggunakan huruf A hingga F. Sistem bilangan ini digunakan untuk menampilkan nilai alamat memori dalam pemrograman komputer.
/subsection {contoh perkalian biner}
Perkalian dengan 3:
Dari Tabel 1 dapat ditentukan
Satuan Hasil Perkalian dengan 3 (SHP3)
k 0 2 4 6 8
SHP3 0 6 2 8 4
Jika k genap maka:
SHP3 = Nilai satuan pada : 2 (10- k)
k 1 3 5 7 9
SHP3 3 9 5 1 7
Jika k ganjil maka:
SHP3 = Nilai satuan pada : 2 (10- k) + 5
Sehingga cara mudah menentukan hasil
perkalian bilangan n digit dengan 3 :
1. Untuk angka terkanan =
Nilai satuan pada : 2 (10- k), k genap
Nilai satuan pada : 2 (10- k) + 5,
k ganjil
Jika memuat puluhan simpan sebagai
”simpanan”
2. Untuk angka di sebelah kirinya =
Nilai satuan pada : 2 (9- k), k genap
Nilai satuan pada : 2 (9- k) + 5,
k ganjil, ditambah ”s” dari
Jurnal Matematika Vol. 11, No.1, April 2008:38-42
40
tetangganya.
Jika dari langkah 1 diperoleh
“simpanan” maka “simpana’ yang ada
ditambahkan pula.
Jika hasilnya memuat puluhan simpan
sebagai ”simpanan”
3. Ulangi langkah 2 sampai digit ke n
4. Untuk digit ke (n+1) =
”s” dari digit ke n + ”simpanan”
dikurangi 2
Dengan demikian jika bilangan yang
dikalikan n digit diperlukan (n+1) langkah
Contoh:
( 1 ) 9876 X 3 = ?
Penyelesaian : Pandang 09876
Langkah 1 : 2 (10-6) = 2 (4) = 8
Langkah 2 : 2 (9-7)+5+”s” dari 6
= 2 (2)+5+3=4+5+3 = 12
= 2 simpan 1
Langkah 3 : 2 (9-8)+
”s” dari 7+”simpanan”
= 2(1)+3+1=2+3+1= 6
Langkah 4 : 2 (9-9)+5+”s” dari 8
= 2 (0)+5+4=0+5+4 = 9
Langkah 5 : “s” dari 9 – 2 = 4-2 = 2
Maka : 9876 X 3 = 29628.
Atau dikerjakan dengan cara lain :
( 2 ) 41692573 X 3 = ?
C = ”simpanan”, H = hasil
k OPERASI C H
3 2(10-3)+5 = 19 1 9
7 2(9-7)+5+1+1 1 1
5 2(9-5)+5+3+1 1 7
2 2(9-2)+2+1 1 7
9 2(9-9)+5+1+1 0 7
6 2(9-6)+4+0 1 0
1 2(9-1)+5+3+1 2 5
4 2(9-4)+0+2 1 2
0 2+1-2 0 1
Jadi 41692573 X 3 = 125077719
