\section
Bit dan Byte memiliki arti istilah yang sering kita dengar atau temukan ketika berurusan dengan komputer atau internet.Sebutan yang seperti ini 
sering sekali biasanya dapat membuat kita menjadi bingung dan linglung. Bit merupakan kependekan dari istilah “Binary Digit” yang memiliki arti 
digit bener.Binary digit adalah satuan-satuan terkecil dalam komputasi digital. Komputer tidak menggunakan angka desimal  dalam menyimpan data nya. Semua data komputer yang sudah ada akan disimpan dalam sebuah angka – angka biner. Dan hanya dua nilai yang bisa dinyatakan 1 bit, yaitu 0 maupun nilai 1, dalam telekuminkasi digital juga seperti itu, semua level tegangan diubah menjadi bentuk data biner.
Sedangkan byte adalah satuan informasi dalam computer yang lebih besar dari bit. Istilah “Byte” pertama diciptakan oleh Dr. Werner Buccholz di tahun 1956, saat itu ia bekerja sebagai seorang ilmuan di IBM. 
Cara membedakan bit dengan Byte adalah dengan mengingat bahwa  huruf “b” kecil untuk bit yang artinya lebih kecil dari Byte, sedangkan “B” besar untuk Byte arinya niainyalebih besar dari bit.
di dalam media penyimpanan itu seperti hardisk,flashdisk, compack disk (CD) atau memory card, kita semua mengenal istilah atau satuan untuk menyebutkan ukuran atau kapasitas dari media penyimpanan, seperti kilo byte, mega byte, giga byte dan tera byte. Dan jika kita ingin mengetahui sebuah informasi suatu ukuran file ( document, photo, video, dan lain-lain
Hardisk maupun flashdisk biasanya akan dimunculkan dalam sebuah satuan Kilo Byte (KB), Mega Byte (MB), Giga Byte (GB), TeraByte (TB), Bytes ataupun yang terkecil dimunculkan dalam satuan Bit. Biasanya untuk file – file yang berukuran kecil atau kurang dari satu Mega Byte (1MB) akan ditampilkan dalam satuan Kilo Byte (KB).
Ada 12 macam satuan dalam Byte, yang diantaranya adalah :

1 Bit = BinaryDigit
8 Bit = 1 Byte
1024 Bytes = 1 KiloByte (KB)
1024 Kilo Bytes = 1 MegaByte (MB)
1024 Mega Bytes = 1 GigaByte (GB)
1024 Giga Bytes = 1 TeraByte (TB)
1024 Tera Bytes = 1 PetaByte (PB)
1024 Peta Bytes = 1 ExaByte (EB)
1024 Exa Bytes = 1 ZettaByte (ZT)
1024 Zetta Bytes = 1 YottaByte (YT)
1024 Yotta Bytes = 1 BrontoByte (BB)
1024 Bronto Byte= 1 GeoByte

Pada artikel ini kami akan membahas 4 satuan yang diantaranya adalah Kilo Byte (KB), Mega Byte (MB), Giga Bytes (GB), Tera Bytes (TB), Peta Byte (PB) dan cara untuk mengkonversikannya. Contoh dari Kilo ke Mega atau dari Mega ke Giga. 
Sebelum itu mari kita bulatkan angka 1024 menjadi 1000 agar lebih mudah dikonversi.
\subsection
A.	Cara Mengkonversi dari Kilo Byte menjadi Mega Byte, Giga Byte dan Tera Byte
1.	Kilo Byte menjadi Mega Byte
Jika kita memiliki 1000 Kb maka akan menjadi 1 Mb. Karena rumusnya adalah :
Kb : 1000 = Mb
Contoh soal :
Dezha memiliki suatu hardisk berukuran 200000 Kb, Bila Dezha Mengkonversikannya ke dalam Mega Byte maka menghitungnya dengan cara :
200000 Kb : 1000 = 200 Mb
Yang berarti hardisk deza berukuran 200 Mb