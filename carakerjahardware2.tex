Dalam sebuah sistem komputer terdapat perangkat keras(Hardware), perangkat keras (Hardware) didefinisikan sebagai komponen-komponen komputer yang dapat ditangkap dengan indra peraba kita. Hardware dalam sistem komputer dibagi menjadi dalam beberapa bagian diantaranya adalah : 1. Perangkat Input 2. Perangkat Proses 3. Perangkat output. Perangkat Input atau output sering dikenal dengan istilah I/O device atau Input / Output Device. I/O device ini adalah perangkat-perangkat komputer yang digunakan untuk masukan dan keluaran. I/O device ini bisa terdapat di dalam atau di luar CPU. Perangkat yang terdapat di luar CPU dikenal dengan istilah Periferal l. Jadi saya yakin contohnya sudah bisa kalian tebak dan sebutkan tentunya.
\section{Cara Kerja Hardware}
Perangkat yang berada di luar CPU diantaranya adalah Keyboard, mouse, monitor ataupun printer. Sedangkan perangkat yang terdapat dalam CPU dikenal dengan istilah Storage Device. Contoh storage device ini seperti Hardisk, CD Room, Disk Drive dan lain sebagainya. Di dalam CPU terdapat CU atau Control Unit, RAM dan ROM. Control unit ada juga yang namanya ALU atau Aritmatic Logical Unit yang berfungsi untuk melakukan berbagai kegiatan yang terkait dengan perhitungan-perhitungan yang dilakukan.
Keyboard Mouse Monitor Printer CPU (Central Processing Unit)/ Perangkat Proses PERANGKAT INPUT/OUTPUT Keyboard ini adalah merupakan alat yang banyak digunakan dan menjadi mutlak untuk di gunakan. Keyboard memiliki fungsi yang mirip dengan mesin ketik pada zaman dahulu. Akan tetapi keyboard ini memiliki  suatu kemampuan lebih yang tidak dimiliki oleh mesin tik pada zaman dulu diantaranya dapat ditemui tombol-tombol fungsi mulai dari F1 sampai dengan F12 yang umumnya digunakan untuk memberikan suatu perintah yang diberikan namun perintah tersebut tergantung daripada aplikasi atau program yang akan digunakan. Keyboard yang selama ini kita gunakan biasanya terdiri atas 2 jenis yakni Keyboard QWERTY dan jenis keyboard DVORAK. Namun keyboard yang sering digunakan dan banyak digunakan saat ini adalah keyboard jenis QWERTY karena lebih mudah digunakan dibandingkan dengan keyboard DVORAK. Dengan pertumbuhan teknologi yang amat pesat membuat keyboard pada masa ini berkembang sangat maju contohnya pada saat ini ada keyboard yang menggunakan wireless system atau tanpa kabel . Struktur-struktur tombol pada keyboard Dari sisi tombol yang digunakan, keyboard memiliki perkembangan yang tidak     terlalu pesat sejak ditemukan pertama kali. Yang terjadi hanyalah penambahan�penambahan beberapa tombol bantu yang lebih mempercepat pembukaan aplikasi program. Secara umum, struktur tombol pada keyboard terbagi atas 4 (empat) , yaitu: � Tombol Ketik (typing keys) Tombol ketik adalah salah satu bagian dari keyboard yang berisi huruf dan angka serta tanda baca. Secara umum, ada 2 jenis susunan huruf pada keyboard, yaitu tipe QWERTY dan DVORAK. Namun, yang terbanyak digunakan sampai saat ini adalah susunan QWERTY. � Numerickeypad adalah bagian khusus dari keyboard yang berisi angka dan berfungsi untuk memasukkan file berupa angka-angka dan operasi perhitungan.Struktur-struktur angkanya disusun menyerupai kalkulator dan alat hitung lainnya. � Tombol Fungsi (Function Keys) Tahun 1986, IBM menambahkan beberapa tombol fungsi pada keyboard standard. Tombol ini dapat dipergunakan sebagai perintah khusus yang disertakan pada sistem operasi maupun aplikasi. � Tombol kontrol (Control keys) Tombol ini menyediakan kontrol terhadap kursor dan layar. Tombol-tombol yang termasuk dalam kategori ini adalah 4 tombol bersimbol panah di antara tombol ketik dan numeric keypad, home, end, insert, delete, page up, page down, control (ctrl), alternate (alt) dan escape (esc). MOUSE Mouse ini adalah sebuah alat yang digunakan sebagai pengatur posisi kursor (tanda panah yang sering kali bergerak ketika kita menggeserkan mouse). Pada awalnya mouse yang ada adalah masih memakai roda di bawahnya, namun dengan perkembangan yang pesat dari tehnologi saat ini mengakibatkan perkembangan perangkat komputer mengalami kemajuan yang luar biasa, saat ini mouse sudah menggunakan tehnologi infrared dan tehnologi wireless. SCANNER Scanner adalah alat yang digunakan secara otomatis untuk memasukkan data baik berupa huruf, 